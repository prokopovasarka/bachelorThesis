\section{Ukázky implementace}
V následujících řádcích jsou prezentovány významné části implementace vyvíjené aplikace, jejich popis spolu
s ukázkami zdrojových kódů.

Implementace se řídí architekturou stanovenou v návrhu a splňuje funkční i nefunkční požadavky uvedené při analýze. 
Sekce je tedy dle využité Component-based architektury rozdělená na popis jednotlivých komponent důležitých pro fungování celé 
aplikace.

\subsection{App}
Komponenta \texttt{App} představuje vstupní komponentu celé vyvíjené aplikace. Hlavním cílem je inicializace navigace a umožňuje 
obrazovkám mít přístup k navigaci. Je nutná pro fungování knihovny \texttt{@react-navigation}.

\begin{verbatim}
<NavigationContainer>
      <RootStack />
</NavigationContainer>
\end{verbatim}

\subsection{RootStack a HomeScreen}
\texttt{RootStack} definuje navigaci celé aplikace pomocí \texttt{createNativeStackNavigator}. Výchozí obrazovkou je \textbf{Home}, 
další obrazovky jsou Camera, Library, Help a ColorDetail. Navigace je vytvořena formou zásobníku \texttt{Stack.Navigator}, každá obrazovka
má vlastní hlavičku. \texttt{HomeScreen} kontroluje, zda se jedná o první spuštění aplikace, či zda byla aplikace v minulosti
již spuštěna, dle této podmínky přesměruje uživatele na \texttt{Help} či \texttt{HomeScreen}.

\begin{verbatim}
React.useEffect(() => {
    const checkFirstLaunch = async () => {
      const alreadyLaunched = await AsyncStorage.getItem('alreadyLaunched');
      if (alreadyLaunched === null) {
        await AsyncStorage.setItem('alreadyLaunched', 'true');
        navigation.replace('Help'); 
      }
    };
    checkFirstLaunch();
}, []);
\end{verbatim}

Obrázek \ref{fig:HomeScreen} zobrazuje úvodní stránku \texttt{HomeScreen} se zobrazením navigace a 
vzhled obrazovky s hlavičkou \texttt{header} definovanou pro všechny obrazovky v \texttt{RootStack}.

\begin{figure}[!ht]
    \centering
    \includegraphics[width=0.7\linewidth]{images/implementation_1.jpg}
    \caption{Ukázka implementace úvodní obrazovky \texttt{HomeScreen} a knihovny \texttt{Library}.}
    \label{fig:HomeScreen}
\end{figure}

\subsection{CameraScreen}
Tato část zajišťuje výběr barev z prostoru a jejich vizualizaci v prostoru. Má na starost práci s kamerou, galerií
a perzistentním ukládáním dat. CameraScreen je hlavní obrazovka kamery. Řídí zároveň komunikaci s AR scénou.

Funkce \texttt{handleTap} zjišťuje pozici kliknutí uživatele, následně se snaží z detekovat barvu pixelu, buď na základě vybrané fotografie
z galerie nebo pomocí \texttt{CaptureRef}, která vytvoří dočasný snímek kamery pro získání pixelu. Díky knihovně
\texttt{react-native-pixel-color} lze získat HEX kód barvy. Následně aktualizuje status.

\begin{verbatim}
const handleTap = async (e: any) => {
    const px = PixelRatio.get();
    const x = Math.round(e.nativeEvent.locationX * px);
    const y = Math.round(e.nativeEvent.locationY * px);
    setTapPos({ x: e.nativeEvent.locationX, y: e.nativeEvent.locationY });

    try {
      const tag = findNodeHandle(viewRef.current);
      if (!tag) throw new Error("View ref not found");
      const uri = capturedPhoto || (await captureRef(tag, { format: "png", quality: 1 }));
      const color = await PixelColor.getHex(uri, { x, y });
      await placeAtPointRef.current?.(x, y, color);

      setSelectedColor(color);
      setStatus("Color selected");
    } catch (err) {
      console.warn("Pixel read error:", err);
      setStatus("Pixel read failed");
    }
};
\end{verbatim}

\texttt{CameraScreen} se stará taktéž o vytvoření a uložení fotografie. V případě snímání kamery funkce \texttt{takePhoto} vytvoří pomocí
\texttt{captureRef} fotografii, pro odstranění UI na fotografii je využit časovač \texttt{setTimeout(r, 100)} a stav \texttt{uiVisible}.

K uložení fotografie slouží funkce \texttt{savePhoto} využívající nativní knihovnu \texttt{@react-native-camera-roll/camera-roll}, která zajišťuje
přístup do knihovny, kam se fotografie ukládá.

\begin{verbatim}
await CameraRoll.saveAsset(capturedPhoto, {
        type: "photo",
        album: "ColorFinder",
});
\end{verbatim}

Poslední význačnou částí CameraScreen je možnost výběru fotografie z galerie. Tu zajišťuje funkce \texttt{pickFromGallery} s \texttt{launchImageLibrary} z knihovny
\texttt{react-native-image-picker}. Pro načtení využívá časovač, během kterého se nastaví vybraná fotografie.

\begin{verbatim}
const pickFromGallery = async () => {
    try {
      const result = await launchImageLibrary({ mediaType: "photo" });
      if (result.didCancel || !result.assets?.[0]?.uri) return;
      setCapturedPhoto(result.assets[0].uri);
      setStatus("Photo loaded");
      setSelectedColor("");
      setUiVisible(false);

      setTimeout(async () => {
        try {
          const tag = findNodeHandle(viewRef.current);
          if (!tag) return;
          const uri = await captureRef(tag, { format: "png", quality: 1 });
          setCapturedPhoto(uri);
        } catch (e) {
          console.warn("Snapshot after gallery load failed:", e);
        } finally {
          setUiVisible(true);
        }
      }, 500);
    } catch {
      Alert.alert("Error", "Failed to pick image");
    }
};
\end{verbatim}

\subsection{SceneAR}
CameraScreen komunikuje se SceneAR pomocí \texttt{ViroARSceneNavigator}. SceneAR zajišťuje logiku rozšířené reality,
tvorbu materiálů a umístění informačního boxu s detekovanou barvou do prostoru.


\subsection{ColorDetail}

\subsection{ColorData}

\subsection{Help}

\subsection{Library}
