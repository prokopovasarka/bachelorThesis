\section{Statická analýza}
Statická analýza kódu představuje důležitou součást procesu vývoje softwaru, jejímž cílem je odhalení 
potenciálních chyb, nekonzistencí a porušení definovaných pravidel již během vývoje aplikace, a to bez nutnosti jejího spuštění. 
Umožňuje včasnou identifikaci problémů, které by se jinak projevily až v pozdějších fázích vývoje nebo při provozu aplikace, a 
přispívá tak ke zvýšení kvality, čitelnosti a dlouhodobé udržitelnosti zdrojového kódu. V této práci je statická analýza integrována do procesu 
\textit{Continuous Integration (CI)}, kde je automaticky prováděna při testování aplikace. Pro kontrolu dodržování standardů a odhalování běžných chyb je využit nástroj ESLint.

\subsection{ESLint}
ESLint je open-source projekt pomáhající nalézt a opravit problémy v JavaScript kódu. S ESLint je možné analyzovat JavaScript kód psaný v prohlížeči i na serveru, v libovolném Frameworku i bez něj. 
Soustředí se na různé typy problémů, od potenciálních chyb běhu, přes nedodržování osvědčených postupů, po problémy se stylem.

Pravidla \textbf{(Rules)} jsou základním stavebním kamenem ESLint. Každé pravidlo ověřuje, zda kód splňuje určitá očekávání a pokud tato očekávání nesplňuje, pomáhá s dalším postupem. Mohou také obsahovat 
další konfigurační možnosti specifické pro nějaké pravidlo. ESLint disponuje knihovnou stanovených pravidel, která analyzuje, lze však vytvářet vlastní pravidla či využívat pravidla vytvořená dalšími vývojáři.
Při vyvíjení softwaru bylo využité základní nastavení ESLint bez přidaných pravidel a v CI pipeline je označené jako \textit{optional}, jelikož hlavní prioritou CI je testování a sestavení aplikace. Avšak tímto způsobem umožňuje 
vylepšovat kvalitu kódu a automatizovat tento proces.~\cite{coursera_github}

\subsection{Dokumentace}
Nedílnou součástí kvalitního softwaru je spolu s automatizovaným testováním a statickou analýzou také dokumentace a přehledná struktura zdrojového kódu. 
Tyto aspekty významně přispívají k dlouhodobé udržitelnosti aplikace a usnadňují její další rozšiřování, pro samotného autora i případné spolupracovníky.
Základní přehled o projektu poskytuje soubor \texttt{README}, který obsahuje shrnutí účelu aplikace a popis pro její sestavení a spuštění. Slouží jako vstupní bod pro 
nové uživatele i vývojáře.

Dalším důležitým prvkem jsou komentáře přímo ve zdrojovém kódu, které vysvětlují složitější části implementace, netriviální algoritmy či specifické chování aplikace. 
Komentáře nejsou využívány k popisu samozřejmých konstrukcí, ale zaměřují se na objasnění důvodů zvoleného řešení. Tyto formy dokumentace jsou v této práci provázány 
se statickou analýzou kódu, která prostřednictvím nástroje ESLint napomáhá udržovat konzistentní styl, čitelnost a dodržování definovaných pravidel napříč celým projektem.
