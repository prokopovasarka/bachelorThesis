\section{Continuous Integration}
Jedním z klíčových kroků při implementaci je zavedení \textit{Continuous integration} (CI). Jedná se o softwarový
postup zahrnující sloučení kódu na sdíleném úložišti a automatické spouštění spolu s testováním. Vztahuje se nejčastěji k fázi sestavení nebo
při integrační fázi. Hlavním cílem CI je rychlejší vyhledávání chyb a jejich snazší řešení, zlepšení kvality kódu a šetření času
při ověřování a vydávání nových aktualizací.~\cite{aws_continuous_integration}

Pro zavedení CI do projektu bylo zvolené úložiště GitHub a proces jeho zavedení je popsaný v následující podeskci.

\section{GitHub}
GitHub je webová platforma pro vývojáře k ukládání a spravování kódu. Umožňuje spolupráci na projektech 
a sdílení open-source projektů~\cite{coursera_github}. Pro vytvoření vlastních workflow pro CI/CD nyní GitHub nabízí 
\textbf{GitHub Actions} umožňující sestavení kódu přímo v repozitáři a spouštění testů. Pro veřejné repozitáře využívající \textit{Github-hosted runners}, což 
je virtuální prostředí poskytnuté od GitHub, je tato funkcionalita bez poplatku. Pro soukromé repozitáře je služba zpoplatněna po využití stanoveného počtu minut.
Poslední možností je spouštět pipeline lokálně, pak je služba taktéž bez poplatku.

Následující popis fungováni workflow v GitHub Actions vychází z~\cite{github_actions_docs}.

Workflow je konfigurovatelný automatizovaný proces, při němž se spouští jeden či více procesů. Je definovaný souborem YAML uloženým 
v repozitáři a spustí se buď dle definovaného plánu, ručně či po spuštění akce v repozitáři. Workflows jsou definované v repozitáři 
v adresáři \texttt{.github/workflows}, v této práci přesněji v souboru \texttt{ci.yml}.

Event je specifická aktivita v repozitáři, která spustí provedení Workflow. Je zadefinována v části \texttt{on}, a to při
\texttt{push} či \texttt{pull\_request} pro vybrané větve.

Job je sada kroků (Steps), které se provedou na stejném runner. Buď se jedná o shell skript či akce (Actions), které budou provedeny. Kroky se prováději ve 
stanoveném pořadí a jsou na sobě vzájemně závislé. V případě této práce se nejprve provádí testy, po kterých následuje sestavení aplikace pro různé platformy. 
Mohou být zároveň nepovinné, tedy pokud nedojde k jejich úspěšnému dokončení, lze i přesto pokračovat.

Action je předdefinovaná sada úloh či kód, který je opakovaně využitelný a provádí úkony v rámci pracovního postupu. Snižuje množství opakujícího se kódu, příkladem 
je načtení repozitáře, nastavení autentizace pro poskytovatele cloud služeb či nastavení správných toolchainů.

Runner je server, na kterém je spuštěna Workflow. GitHub nabízí Ubuntu Linux, Microsoft Windows i macOS runner a každý může provádět jeden Job v daný moment. 
Ke spouštění testů je využit \texttt{ubuntu-latest} a k sestavení iOS aplikace \texttt{macos-latest}.

Pro potřeby této bakalářské práce je využita CI, dalším krokem by mohl být \textit{Continuous deployment} sloužící k automatizaci nasazení a publikování projektů, například 
přímo do obchodu s aplikacemi. Pro sdílení aplikace v AppStore je však vyžadována placená služba Apple Developer Program a aplikace musí projít schvalovacím procesem.

