Hlavním cílém této práce je vývoj multiplatformní aplikace pro detekci barev a tvorbu barevných schémat
za použití programovacího jazyka React native, a to dle tradičních postupů softwarového inženýrství. Práce se tedy soustředí
na analýzu, sestavení požadavků, návrh, implementaci a následné testování.

Prvním bodem je analýza teorie barev, vnímání barev a popsání známých barevných sad a tvorby palet.
Současně se analýza zabývá extrakcí informací z fotografií, a nakonec také využitím AR pro práci s 
mobilním zařízením a fotografiemi.

Další část se budě věnovat rozborem existujících řešení, které detekují barvy z fotografií či tvoří barevná schémata 
na základě stanovených barev.

Dále proběhne sestavení požadavků a případů užití výsledné aplikace. Bude následovat návrh aplikace, tvorba architektury a příprava konceptu
uživatelského rozhraní. Dle tohoto návrhu proběhne implementace aplikace a následné testování.

Nakonec proběhne vyhodnocení výsledné aplikace spolu s dílčími cíli této práce, bude prodiskutován taktéž přínos a případný možný rozvoj aplikace.