Barvy a vnímání barev je i přes její důkladné zkoumání v historii stále velmi subjektivní záležitostí. 
Přesto se v průběhu času některým teoretikům, vědcům a filozofům podařilo stanovit určitá pravidla a vlastnosti barev,
které se ve společnosti hojně využívají. Ať už se jedná o předpokládaný umělecký průmysl, v tomto smyslu 
nejen malba či kresba, využití v psychologii, digitálním odvětví, marketingu, či dokonce politice, barvy jsou 
důležitým aspektem, díky kterému můžeme v lidech probouzet emoce, prohlubovat naše myšlenky, v určitých případech 
je využívat také jako nástroj manipulace.

Barvy působí v nervovém systému. Ne každý však má možnost barvy vidět. Je tedy otázkou, jak se takový člověk může 
přizpůsobit dnešní společnosti a zda je možné od společnosti takovému člověku porozumět.

Hlavním cílem této bakalářské práce je proto vyvinout multiplatformní aplikaci, která s pomocí rozšířené reality umožní
rozeznávat barvy objektů v reálném světě a vytvářet další barevné palety. Díky této aplikaci mohou lidé s poruchou vnímání barev
rozlišovat barvy a lépe chápat návaznost na další odstíny. Zároveň je aplikace pomocníkem, díky kterému mohou umělci, tvůrci obsahu
a designéři rychle detekovat barvy a vytvářet soudržná schémata.

Již existující řešení se soustředí většinou pouze na jediný aspekt, buď pouhé rozeznávání barev na fotografii či tvorbu palet na základě stanovené barvy.
Pro uživatele právě sjednocení těchto funkcionalit umožňuje správné porozumění barvám a jejich vnímání. Práce se zabývá analýzou, návrhem a implementací 
aplikace podle tradičních postupů softwarového inženýrství.

