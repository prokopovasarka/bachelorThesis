\section{Možný rozvoj aplikace}
Tato sekce popisuje další možný rozvoj vytvořené mobilní aplikace v budoucnu. Je rozdělena
do dvou podsekcí, první se týká možnosti rozvoje vyplývající z testování, druhá část vyplývá z požadavků.

\subsection{Rozvoj na základě výsledků testování}
Díky uživatelskému testovány vyplynulo několik užitečných návrhů pro budoucí možný rozvoj aplikace.
Jedním z návrhů byla větší interakce s kamerou, možnost zoomu či podpora širokoúhlého objektivu. Zároveň 
byl zmíněn návrh pro vytvoření filtrů, které by simulovaly poruchy barvného vidění pro lepší pochopení problematiky.

V rámci dalšího rozvoje by mohlo dojít taktéž k lepší implementaci nápovědy, například interaktivním způsobem jako překryv obrazovky s šipkami, kam 
je třeba kliknout pro danou akci.

Tyto návrhy nemění fungování samotné aplikace, avšak jejich implementace by mohla vylepšit celkový zážitek z jejího používání.

\subsection{Rozvoj na základě požadavků}
Vhodným adeptem pro další rozvoj aplikace jsou neimplementované funkční požadavky. Jedná se o požadavky s prioritou \textit{will not have} a 
jeden s prioritou \textit{could have}. Jelikož se jedná o požadavky, které plynule navazují na již implementované funkcionality, bylo by vhodné
, i díky jednoduchosti integrace, implementovat je jako jedny z prvních.