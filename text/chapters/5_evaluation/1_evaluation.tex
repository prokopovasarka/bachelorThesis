\section{Výsledky práce}
Následující sekce se věnuje celkovému zhodnocení výsledků jednotlivých fází vývoje po implementaci a testování.
Zároveň poskytuje vyhodnocení plnění funkčních a nefunkčních požadavků a porovnání s existujícími aplikacemi.

\subsection{Shrnutí výsledků}
Analytická část poskytla důkladnější vhled do teorie barev spolu s popisem barevných systému.
Část byla věnována taktéž problematice barevného vidění a poruchám barevného vidění, následované 
rešerší existujících aplikací včetně zaznamenání jejich výhod a nevýhod. Díky analytické části bylo možné 
sestavit funkční a nefunkční požadavky pro vyvíjenou aplikaci.

Vybraný \textit{Component-based} návrhový vzor se osvědčil při vývoji a poskytl díky rozdělení do komponent 
lepší organizaci a přehlednost celého kódu. Díky této volbě lze aplikaci v budoucnosti dále rozšiřovat, 
umožnil taktéž snadné testování. 

Volba VSCode spolu s Xcode pro spouštění aplikace se ukázalo jako vhodným kompromisem. VSCode poskytuje přehlednější prostředí a více 
možností přizpůsobení. Vzhledem k tomu, že bylo nutné při implementaci spouštět aplikaci na fyzickém telefonu, jelikož simulátor nepovoluje rozšířenou realitu,
Xcode sloužil jako funkční prostředník při sestavení aplikace pro iPhone.

Implementační část sloužila k vývoji aplikace, aby následně mohla být řádně otestována a mohly být vyladěny její nedostatky.

Výsledkem bakalářské práce je aplikace pro platformu iOS i Android detekující barvy za využití rozšířené reality, celý proces vývoje tak lze 
považovat za úspěšný.

\subsection{Splnění požadavků}
Výsledná aplikace splňuje veškeré nefunkční požadavky stanovené v analýze. Jedná se o multiplatformní aplikaci pro Android i iOS, vhodná pro modely Apple iPhone 8 a Android 7.0 a vyšší, využívá rozšířenou realitu
k detekci barev. Aplikace zároveň funguje plynule a díky jednoduchému UI a zvoleným UI prvkům je přístupná všeobecnému publiku. Aplikaci 
je možné snadno rozšiřovat o další palety i možnosti interakce.

Veškeré funkční požadavky s prioritou \textit{must have} a \textit{should have} byly splněny. Taktéž došlo z většiny ke splnění požadavků 
označených \textit{could have}. V souladu s časovým rozvržením nebyly implementovány požadavky s prioritou \textit{will not have}.

\subsection{Porovnání s existujícími aplikacemi}
V poslední subsekci dojde k porovnání výsledné aplikace s již existujícími konkurenčními aplikacemi, které byly představeny v analýze.
Jedná se o Color Blind Pal, Color Identifier: Color Picker a Color Name Recognizer.

\paragraph{Color Blind Pal}\mbox{}\\
Aplikace umožňuje rozpoznávat barvy z kamery, tedy má pdobnou úlohu, jako aplikace vyvinutá v této práci. Oproti Color Blind Pal je však
implementována funkce rozšířené reality a díky možnosti volit bod na kameře stiskem se zamezilo automatickému přeskakování mezi barvami na spektru,
jako v Color Blind Pal, které působí chaoticky. Funkcí navíc je také možnost detekce barvy z fotografie a ukládání barev. Uživatel se k nim tak později může 
vrátit.

\paragraph{Color Identifier: Color Picker}\mbox{}\\
Color Identifier: Color Picker disponuje velkou řadou stejných funkcionalit, jako vyvíjená aplikace. Dokáží detekovat barvu z kamery, ukládat si nové barvy a vytvářet barevné palety.
Vyvíjená aplikace v této práci, stejně jako v předchozím porovnání, nabídne navíc AR režim, kterým ani jedna z aplikací nedisponuje. Zároveň je velkou výhodou aplikace vyvíjené v této práci 
její jednoduchost. Color Identifier: Color Picker obsahuje mnoho funkcionalit, avšak toto množství zhoršuje uživatelskou přívětivost a v aplikaci lze nalézt nadbytečné funkce. Jednoduchost a 
snaha soustředit se na nejdůležitější složky softwaru, které může uživatel využívat, umožňuje vytvořit přátelštějští prostředí.

\paragraph{Color Name Recognizer}\mbox{}\\
Color Name Recognizer umožňuje detekci barev z kamery i fotografií. Její výhodou je velká interakce s kamerou a možnost pozastavení kamery pro detekci barvy. Tato funkcionalita byla implementována 
i v aplikaci v této práci. V režimu kamery se však aplikace Color Name Recognizer potýká se stejným problémem, jako v prvním případě, a sice přeskakování barev a chaotičnost. Díky těmto dvěma příkladům
byl při vývoji kladen důraz na větší konzistenci a stabilitu aplikace a bylo navrženo řešení pro lepší uživatelskou přívětivost.   

