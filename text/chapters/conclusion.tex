Cílem této bakalářské práce bylo vytvořit multiplatformní aplikace detekující barvy z kamery 
za pomocí rozšířené reality a vytváření odpovídajících barevných schémat. To zahrnuje provedení analýzy, 
vytvoření návrhu, implementace, testování aplikace a následné vyhodnocení výsledků s návrhem dalšího možného 
rozvoje aplikace.

Prvním krokem byla analýza teorie barev a analýza barevného vidění spolu s popisem poruch barevného vidění. 
Zahrnovala taktéž rešerši konkurenčních aplikací včetně vyhodnocení jejich výhod a nevýhod. Na základě získaných poznatků 
proběhlo sestavení funkčních a nefunkčních požadavků spolu s případy užití. Kapitola byla zakončena sestavením stavového 
diagramu.

V druhé kapitole proběhlo zvolení využívaných technologií a zvážení dalších možných řešení. Následovalo porovnání architektur a návrhových vzorů a došlo ke zvolení vhodné architektury vyvíjené aplikace tak, aby 
vyhověla požadavkům, tedy \textit{Component-based architektura}. Poslední část se věnovala návrhu uživatelského rozhraní a vytvoření wireframů.

Třetí kapitola se již soustředila na samotnou implementaci aplikace, popisem kontinuální integrace, organizací a rozdělením do jednotlivých komponent 
dle funkcionalit. Pro implementaci bylo v návrhu zvoleno IDE VSCode, spolu s Xcode a frameworkem Expo pro sestavení a pro využití rozšířené reality se nejvhodnější volbou stala Viro knihovna. 
Nejvýznačnější části jsou doplněny ukázkami kódu.

Navazující část o testování popisuje unit testování a osvědčené postupy při testování během vývoje. Součástí jsou také ukázky unit testů vyvíjené aplikace.
Následně proběhlo uživatelské testování s reálnými uživateli pro ověření funkčnosti a analýza výsledků testování.
V závěru proběhlo zhodnocení výsledků práce, porovnání implementace s existujícími aplikacemi a na základě výsledků testování a vyhodnocení plnění funkčních požadavků byla kapitola zakončena možným 
rozvojem aplikace v budoucnu.

Tak byly veškeré dílčí cíle a hlavní cíl splněny, tedy došlo i k úspěšnému splnění zadání této bakalářské práce.