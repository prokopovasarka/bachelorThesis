\section{Uživatelské rozhraní}
Poslední část návrhu je věnová návrhu uživatelského rozhraní. Řídí se 10 obecnými zásadami
od Jakoba Nielsena pro návrh interakcí, které stanovují obecné principy založené na problémech, se kterými
se uživatelé setkávají při používání digitálních systémů~\cite{nielsen2024ten}.

Následující podsekce popisují jednotlivé části aplikace doplněné o wireframes, které představují zjednodušený \textbf{přibližný} vzhled
daného uživatelského rozhraní. Návrh poslouží jako předloha pro implementaci.

\subsection{První spuštění}
Po prvním spuštění aplikace se uživatel nachází na úvodní obrazovce s nápovědou, která vysvětluje, jak aplikaci používat a jak se
zorientovat. Je složená z textu s vizuálním doprovodem. Poté lze nápovědu ukončit stisknutím spodního tlačítka,
který uživatele dovede na hlavní obrazovku s menu, viz Obrázek \ref{fig:first_wireframe}.

\begin{figure}[!ht]
    \centering
    \includegraphics[width=0.8\linewidth]{images/first_wireframe.jpg}
    \caption{Wireframe obrazovky s nápovědou při prvním spuštění a hlavního menu.}
    \label{fig:first_wireframe}
\end{figure}

\subsection{Camera}
Z domovské stránky může uživatel pokračovat na stránku s kamerou z hlavního menu, které je záměrně ponecháno minimalistické. Tlačítka jsou velká a jasně označená
textem navigace. V kameře je jasně označené, zda lze tlačítko použít, dle jeho průhlednosti. Vybledlá tlačítka nelze stisknout. Jsou popsané buď slovně dle akce, kterou uživatel 
může provést, nebo prezentovány pomocí ikon reprezentující akci napříč platformami. Vzhled je volen na základě osvědčených postupů a snaží se zachovat 5. pravidlo Nielsenova desatera dle~\cite{nielsen2024ten}, které
popisuje konzistentnost a dodržování konvencí platformy i celého průmyslu. Příkladem je ikona galerie či zpětné tlačítko v levném horním rohu, které uživatele zavede vždy na předchozí stránku.

Při zvolení barvy je vyobrazen zvolený bod spolu s poloprůhledným boxem obsahující čtverec s detekovanou barvou, název a kód barvy. Vzhled boxu je zachován, ať už v případě AR kamery či při 
detekci barvy z fotografie/obrázku z galerie. Pro udržení informovanosti uživatele dle doporučení číslo 1 v~\cite{nielsen2024ten} je v horní části obrazovky připraven text s aktuálním statusem určující, 
zda je aktuálně detekována barva, zda je kamera připravena či probíhá načítání. Ukázku lze vidět na Obrázku \ref{fig:cam_wireframe}

\begin{figure}[!ht]
    \centering
    \includegraphics[width=0.8\linewidth]{images/camera_wireframe.jpg}
    \caption{Ukázka obrazovek při spuštění Camera.}
    \label{fig:cam_wireframe}
\end{figure}

\subsection{Library}
Z hlavní obrazovky je možné přistoupit na stránku s knihovnou uložených barev, jak je vidět na Obrázku \ref{fig:lib_wireframe}. Knihovna vyobrazuje čtverce s definovaným tvarem, je zachováno minimalistické prostředí. Při výběru barev je
zjevné ohraničení zvolených barev. Při odebrání výběru je ohraničení zrušeno. Ve spodní části se nachází dvě tlačítka s jasně popsanou akcí pomocí textu, tedy buď vymazání barev či odebrání označení u všech barev.
Levé zpětné tlačítko je zachováno napříč celou aplikací jednotně.

\begin{figure}[!ht]
    \centering
    \includegraphics[width=0.8\linewidth]{images/library_wireframe.jpg}
    \caption{Ukázka obrazovek v knihovně barev.}
    \label{fig:lib_wireframe}
\end{figure}

\subsection{Detail}
Obrazovka detail je dostupná při kliknutí na barvu v knihovně či při kliknutí na ikonu palety v kameře po zvolení barvy. Detail vyobrazuje barvu spolu se jménem a dalšími
kódy popisující barvu. Ve spodní části se nachází tři palety, které jsou jasně popsané názvem palety a ohraničené pro lepší odlišení od pozadí. (viz Obrázek \ref{fig:detail_wireframe})

\begin{figure}[!ht]
    \centering
    \includegraphics[width=0.4\linewidth]{images/detail_wireframe.jpg}
    \caption{Ukázka obrazovky vyobrazující detail o barvě.}
    \label{fig:detail_wireframe}
\end{figure}

\subsection{Help}
Kromě prvního spuštění je možné vstoupit do obrazovky s nápovědou z hlavní obrazovky. Struktura je zachována pro dodržení konzistence, text je doprovázen vizuály pro vyšší atraktivitu
a zřejmost informací. Spodní tlačítko je nahrazeno zpětným tlačítkem v levém horním rohu, stejně jako je tomu ve zbytku aplikace. Wireframe obrazovky je vyobrazen na Obrázku~\ref{fig:help_wireframe}.

\begin{figure}[!ht]
    \centering
    \includegraphics[width=0.8\linewidth]{images/help_wireframe.jpg}
    \caption{Ukázka obrazovky při spuštění nápovědy.}
    \label{fig:help_wireframe}
\end{figure}

\subsection{Shrnutí návrhu uživatelského rozhraní}
Uživatelské rozhraní aplikace zachovává konzistenci napříč všemi obrazovkami, má jednotný design, barvy i fonty. Pomocí
známých a jasně rozpoznatelných ikon a textu jasně popisuje akce, které uživatel může vykonávat, aby zachoval využívané standardy a dle pravidla 6 v~\cite{nielsen2024ten}
bylo minimalizováno využití paměti.
Návrh uživatelského rozhraní pomohl uzavřít kapitolu o návrhu aplikace. Ten zahrnuje také výběr architektury včetně popsání alternativ spolu s porovnáním a výběrem technologií, a to 
za dodržení běžných postupů softwarového inženýrství pro splnění bodů vyplývajících ze zadání této práce. 


