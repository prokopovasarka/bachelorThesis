\section{Uživatelské rozhraní}
Poslední část návrhu je věnová návrhu uživatelského rozhraní. Řídí se 10 obecnými zásadami
od Jakoba Nielsena pro návrh interakcí, které stanovují obecné principy založené na problémech, se kterými
se uživatelé setkávají při používání digitálních systémů~\cite{nielsen2024ten}.

Následující podsekce popisují jednotlivé části aplikace doplněné o wireframes, které představují zjednodušený \textbf{přibližný} vzhled
daného uživatelského rozhraní. Návrh poslouží jako předloha pro implementaci.

\subsection{První spuštění}
\begin{figure}[!ht]
    \centering
    \includegraphics[width=0.9\linewidth]{images/first_wireframe.jpg}
    \caption{Wireframe obrazovky s nápovědou při prvním spuštění a hlavního menu.}
    \label{fig:first_wireframe}
\end{figure}

\subsection{Camera}
\begin{figure}[!ht]
    \centering
    \includegraphics[width=0.9\linewidth]{images/camera_wireframe.jpg}
    \caption{Ukázka obrazovek při spuštění Camera.}
    \label{fig:cam_wireframe}
\end{figure}

\subsection{Library}
\begin{figure}[!ht]
    \centering
    \includegraphics[width=0.9\linewidth]{images/library_wireframe.jpg}
    \caption{Ukázka obrazovek v knihovně barev.}
    \label{fig:lib_wireframe}
\end{figure}

\subsection{Detail}
\begin{figure}[!ht]
    \centering
    \includegraphics[width=0.5\linewidth]{images/detail_wireframe.jpg}
    \caption{Ukázka obrazovky vyobrazující detail o barvě.}
    \label{fig:detail_wireframe}
\end{figure}

\subsection{Help}
\begin{figure}[!ht]
    \centering
    \includegraphics[width=0.9\linewidth]{images/help_wireframe.jpg}
    \caption{Ukázka obrazovky při spuštění nápovědy.}
    \label{fig:help_wireframe}
\end{figure}



