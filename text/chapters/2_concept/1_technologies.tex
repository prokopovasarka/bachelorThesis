\section{Technologie}
Cílem této sekce je výběr technologií pro úspěšný vývoj aplikace. V první části budou přiblíženy rozdíly multiplatformního a 
nativního vývoje, následně dojde k porovnání vhodných frameworků a odůvodnění finální volby technologie pro implementaci aplikace.

% react native, Viro AR, Expo
\subsection{Nativní vs multiplatformní vývoj}
Následující přehled shrnuje základní rozdíly mezi nativním a multiplatformním vývojem mobilních aplikací dle~\cite{argo22_nativni_vs_multiplatformni}.
Nativní aplikace jsou vyvíjeny pro konkrétní platformu, typicky pro iOS od firmy Apple či Google Android. V závislosti na
zvolené platformě je volen také specifický programovací jazyk. Pro iOS lze využít například Objective-C či Swift, pro Android je
využívána Java či Kotlin.

Mezi výhody vývoje nativní aplikace patří vysoký výkon a plynulost způsobené optimalizací pro danou platformu. Zároveň je jednodušší
přístup k nativním funkcím, jakými jsou například specifická nastavení či GPS, a díky přízpůsobení UX/UI konkrétní platformě stoupá spokojenost uživatelů.

Nevýhody vývoje nativních aplikací jsou primárně vyšší náklady a delší vývoj vzhledem k nutnosti vývoje samostatných verzí pro platformy. Zároveň
jsou vyšší nároky na údržbu a aktualizaci verzí.

Cíl multiplatformních aplikací je vývoj pro více platforem najednou za využití jednoho frameworku. Cílem je hlavně rychlejší vytváření aplikací,
významné snížení nákladů a možnost škálování. Využívají se moderní frameworky, jako například React Native, které dosahují vysokého výkonu i přívětivého
uživatelského zážitku napříč platformami.

Avšak i multiplatformní vývoj musí čelit různým problémům. Mezi nevýhody patří omezený přístup k některým nativním funkcím, možné problémy s kompatibilitou či
nižší výkon u graficky náročných aplikací.

V této bakalářské práci je vyvíjena multiplatformní aplikace. Hlavními důvody jsou větší možnosti testování i rozšiřitelnosti a velikost prostoru pro design. Aplikace taktéž
vyžaduje přístup pouze k těm nativním funkcím, které jsou jednoduše přístupné na obou platformách, což výrazně ulehčuje její vývoj.

\subsection{Vývojové prostředí}
Při vývoji multiplatformní aplikace, tedy aplikace dostupné pro systémy iOS i Android, případně také pro desktopové aplikace, je možné
využít různá vývojová prostředí, taktéž neexistuje omezení pro operační systém. Je tedy možné aplikace vyvíjet na MacOS, Windows či Linux systémech. 
Příkladem může být Visual Studio Code, Microsoft Visual Studio, Eclipse či Xcode.

Přestože je vývoj multiplatformní aplikace obecně nezávislý na cílovém operačním systému, pro účely
testování a ladění aplikace na zařízení se systémem iOS je nutné využít vývojové prostředí Xcode, jenž je vyžadováno
platformou iOS. Ačkoliv samotný vývoj proběhne v prostředí Visual Studio Code kvůli jeho přívětivosti, Xcode sloužil jako klíčový prvek pro sestavení aplikace k jejímu
úspěšnému spuštění.

Pro vývoj aplikace v Xcode je vyžadováno vlastnictví zařízení Mac s operačním systémem MacOS. Xcode je integrované vývojové prostředí,
které lze bezplatně využívat a nabízí podporu velkého množství programovacích jazyků včetně C, C++, Python a další~\cite{ekren_what_is_xcode}.

\subsection{Výběr frameworku}
K vývoji multiplatformních aplikací existuje několik frameworků nabízejících různé funkcionality, které usnadňují a zrychlují proces vývoje aplikací. Výběr frameworku
může být ovlivněn programovacím jazykem, výběrem knihoven či možnostmi integrace. Mezi nejznámější a nejrozšířenější frameworky patří
Flutter, React Native a Kotlin Multiplatform.

\paragraph{Flutter}\mbox{}\\
Flutter je framework vhodný pro mobilní i desktopové a webové aplikace. Využívá programovací jazyk Dart vytvořený společností Google a podporuje využití AR v aplikacích.
Výhodou frameworku Flutter je možnost zobrazení modifikace aplikace bez nutnosti rekompilace, podporuje také Material Design systém poskytující různé
komponenty a nástroje následující osvědčené postupy pro tvorbu uživatelského rozhraní.~\cite{kotlin_multiplatform_cross_platform_frameworks}

\paragraph{Kotlin Multiplatform}\mbox{}\\
Kotlin Multiplatform je open-source technologie vytvořená JetBrains kombinující výhody nativního vývoje s možnostmi vývoje multiplatformních aplikací.
Využívá programovací jazyk Kotlin a jeho největší výhodou je možnost využívání kódu napříč platformami, možnost psaní nativního kódu a snadná integrace
do jakéhokoliv projektu~\cite{pavlu_kotlin_multiplatform_stable}. Naproti tomu Kotlin Multiplatform nedisponuje podporou knihoven pro rozšířenou realitu.

\paragraph{React Native}\mbox{}\\
React Native je open-source UI framework vytvořen společností Meta Platforms. Využívá již rozšířený JavaScript/TypeScript a díky Fast Refresh funkci mohou vývojáři
prohlížet změny v aplikaci ihned po vytvoření změny v komponentách. Zároveň se React Native soustředí na uživatelské rozhraní, čímž podporuje responzivní
prostředí, které lze snadno přizpůsobit. Díky rozsáhlé komunitě usnadňuje vývoj aplikací a poskytuje velké množství knihoven včetně knihoven rozšířené reality.~\cite{kotlin_multiplatform_cross_platform_frameworks}

\subsection{Shrnutí výběru frameworku}
Pro vývoj této bakalářské práce je využit framework React Native, a to na základě několika faktorů. Prvním je využitý jazyk JavaScript/TypeScript, jenž je jednoduchý, responsivní
a rozšířený, což usnadňuje celkový vývoj aplikace. Dalším důvodem je podpora knihoven rozšířené reality a rychlé projevení změn při psaní kódu.

\subsection{Expo}
Pro zjednodušení celkového procesu vývoje je využit framework Expo, jenž je definován jako sada nástrojů a služeb postavených na React Native. Slouží
k usnadnění pracovních postupů a poskytuje prostor vývojářům soustředit se na psaní samotného kódu bez nutnosti soustředit se na nastavení konfigurací a nativních nástrojů~\cite{sarwar_react_native_with_expo}.
Expo slouží jako mezikrok mezi samotným vývojem v React Native a následným sestavením aplikace v Xcode.