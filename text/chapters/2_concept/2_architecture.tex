\section{Architektura}
% component-based architektura
Pro návrh struktury systému a specifikaci interakcí mezi jednotlivými částmi je nutné zvolit správnou
architekturu. Ta nám usnadní vývoj a škálovatelnost, zároveň určuje komplexitu našeho projektu. 
Architekturu vybereme na základě požadavků a výběru technologií vhodných pro náš projekt.

Kapitola se zaměřuje na popis známých architektur a návrhových vzorů využívaných pro vývoj multiplatformních
aplikací včetně jejich způsobu komunikace. Následně budou shrnuty jejch výhody a nevýhody. V závěru proběhne porovnání
těchto návrhových vzorů včetně určení nejvhodnějšího výběru pro aplikaci v této práci.

\section{MVC}
MVC, celým názvem "Model-View-Controller", rozděluje aplikaci do tří částí, každá reprezentující
rozdílnou problematiku aplikace. Její popis je vyobrazen níže dle~\cite{tutorialspoint_mvc}.

\paragraph{Model}
Reprezentuje objekt nesoucí data. Může taktéž v sobě uchovávat logiku pro aktualizaci Controller části při aktualizaci
dat.

\paragraph{View}
View vizualizuje data, které model obsahuje.

\paragraph{Controller}
Působí pro obě části, tedy jak pro Model, tak pro View. Stará se o tok dat do objektu a aktualizuje zobrazení, tedy
View část, pokaždé, když dojde ke změně dat. udržuje zároveň separaci mezi Model a View.


\section{MVVM}
Model MVVM obsahuje tři hlavní komponenty: Model, View a ViewModel. ViewModel se stará o izolaci View od Model a umožňuje
vývoj komponenty Model bez závislosti na View. Na vysoké úrovni však funguje komunikace mezi těmito komponentami a vzájemně
se do určité míry ovlivňují. Popis MVVM vychází z~\cite{microsoft_maui_mvvm}.

\paragraph{View}
Zodpovídá za definování struktury a vizuálníá obsah viditelný uživateli. Neobsahuje obchodní logiku, kromě specifických případů,
a lze být reprezentováno šablonou dat definující rozhraní.

\paragraph{ViewModel}
Implementuje příkazy a vlastnosti zajišťující vazbu dat na model View a upozorňuje jej na změnu stavu.
Tyto příkazy definují funkce v uživatelském rozhraní a model View určuje formu jejich zobrazení.

\paragraph{Model}
Model zapouzdržuje data aplikace. Lze si jej představit jako reprezentaci doménového modelu aplikace zahrnující obchodní a ověřovací
logiku zároveň. Je možné spojení s dalšími službami či úložišti zajišťující přístup k datům.

\begin{figure}[!ht]
    \centering
    \includegraphics[width=0.9\linewidth]{images/mvvm-pattern.png}
    \caption{Ukázka komunikace modelu MVVM dle~\cite{microsoft_maui_mvvm}.}
    \label{fig:MVVM}
\end{figure}


\section{Component-based}
Komponentová architektura je založená na opakovaně využitelných částech aplikace, kde každá komponenta má definovanou funkcionalitu, jenž
je vložena do aplikace bez úpravy ostatních komponent. Dle zdroje~\cite{mendix_component_based_architecture} si lze komponentu představit jako kostku lega.

Při stavbě struktury z kostek lega, v tomto případě aplikace, lze vybírat z různých tvarů, velikostí a barev, které představují její komponenty. Každá
kostka má přitom své určení, například dveře, různé konstrukční prvky či okna. Ty lze dle jejich vlastností propojit s ostatními bloky, jejich přídání či 
odebrání má přitom minimální dopad na strukturu. I v případě kompikovanějších programů je zachována alegorie, která popisuje koncept této architektury.

