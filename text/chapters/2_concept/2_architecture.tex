\section{Architektura a architektonické vzory}
% component-based architektura
Pro návrh struktury systému a specifikaci interakcí mezi jednotlivými částmi je nutné zvolit správnou
architekturu a architektonický vzor. Ta nám usnadní vývoj a škálovatelnost, zároveň určuje komplexitu našeho projektu. 
Architekturu spolu s architektonickým vzorem vybereme na základě požadavků a výběru technologií vhodných pro náš projekt.

Kapitola se zaměřuje na popis známých návrhových vzorů využívaných pro vývoj multiplatformních
aplikací včetně jejich způsobu komunikace. Následně budou shrnuty jejich výhody a nevýhody. V závěru proběhne porovnání
těchto návrhových vzorů včetně určení nejvhodnějšího výběru pro aplikaci v této práci.

\subsection{MVC}
MVC, celým názvem "Model-View-Controller", rozděluje aplikaci do tří částí, každá reprezentující
rozdílnou problematiku aplikace. Její popis je vyobrazen níže dle~\cite{tutorialspoint_mvc}.

\paragraph{Model}
Reprezentuje objekt nesoucí data. Může taktéž v sobě uchovávat logiku pro aktualizaci Controller části při aktualizaci
dat.

\paragraph{View}
View vizualizuje data, které model obsahuje.

\paragraph{Controller}
Působí pro obě části, tedy jak pro Model, tak pro View. Stará se o tok dat do objektu a aktualizuje zobrazení, tedy
View část, pokaždé, když dojde ke změně dat. udržuje zároveň separaci mezi Model a View.


\subsection{MVVM}
Model MVVM obsahuje tři hlavní komponenty: Model, View a ViewModel. ViewModel se stará o izolaci View od Model a umožňuje
vývoj komponenty Model bez závislosti na View. Na vysoké úrovni však funguje komunikace mezi těmito komponentami a vzájemně
se do určité míry ovlivňují. Popis MVVM vychází z~\cite{microsoft_maui_mvvm}.

\paragraph{View}
Zodpovídá za definování struktury a vizuální obsah viditelný uživateli. Neobsahuje obchodní logiku, kromě specifických případů,
a lze být reprezentováno šablonou dat definující rozhraní.

\paragraph{ViewModel}
Implementuje příkazy a vlastnosti zajišťující vazbu dat na model View a upozorňuje jej na změnu stavu.
Tyto příkazy definují funkce v uživatelském rozhraní a model View určuje formu jejich zobrazení.

\paragraph{Model}
Model zapouzdržuje data aplikace. Lze si jej představit jako reprezentaci doménového modelu aplikace zahrnující obchodní a ověřovací
logiku zároveň. Je možné spojení s dalšími službami či úložišti zajišťující přístup k datům.

\begin{figure}[!ht]
    \centering
    \includegraphics[width=0.9\linewidth]{images/mvvm-pattern.png}
    \caption{Ukázka komunikace modelu MVVM dle~\cite{microsoft_maui_mvvm}.}
    \label{fig:MVVM}
\end{figure}


\subsection{Component-based}
Component-based je založená na opakovaně využitelných částech aplikace, kde každá komponenta má definovanou funkcionalitu, jenž
je vložena do aplikace bez úpravy ostatních komponent. Dle zdroje~\cite{mendix_component_based_architecture} si lze komponentu představit jako kostku lega.

Při stavbě struktury z kostek lega, v tomto případě aplikace, lze vybírat z různých tvarů, velikostí a barev, které představují její komponenty. Každá
kostka má přitom své určení, například dveře, různé konstrukční prvky či okna. Ty lze dle jejich vlastností propojit s ostatními bloky, jejich přídání či 
odebrání má přitom minimální dopad na strukturu. Vývoj aplikací je sice náročnější proces, avšak i v případě kompikovanějších programů je tato analogie popisující
koncept Component-based vzoru relevantní.

\subsection{Výhody a nevýhody zmíněných architektonických vzorů}
Po stručném popsání jednotlivých návrhových vzorů je třeba před závěrečným výběrem shrnout si
jejich výhody a nevýhody. Nejprve následuje popis \textbf{výhod} plynoucích z využívání jednotlivých vzorů.

\paragraph{MVC}
MVC umožňuje oddělení uživatelského rozhraní od dat. Tím usnadňuje změnu a další vývoj jednotlivých částí, aniž by 
zasahovaly jedna do druhé. Program je tak lépe rozčleněn, funkcionality se navzájem nepřekrývají a je jasně určená zodpovědnost každé z nich.~\cite{shakuro_mvc_vs_mvvm}

\paragraph{MVVM}
Ačkoliv je MVVM velmi podobný MVC, nabízí určité výhody, které MVC ve svém modelu nezakomponovává. MVVM podporuje vazbu dat mezi View a ViewModel,
podporuje vytváření více vztahů mezi nimi a usnadňuje Unit testování. Obchodní logika je navíc zcela oddělená od UI.~\cite{shakuro_mvc_vs_mvvm}

\paragraph{Component-based}
Komponenty lze znovu využívat bez nutnosti jejich změny napříč aplikací, ale také napříč různými projekty. Systémy mohou být jednoduše rozšiřovány a komponenty
snadno modifikovány s minimálním zásahem do celku. Kód je zároveň díky přesnému specifikování každé komponenty organizovanější a umožňuje vytvářet čistou
strukturu.~\cite{mendix_component_based_architecture}
\mbox{}\\\mbox{}\\
Každý model však skýtá i určité nevýhody jeho využití.

\paragraph{MVC}
Nevýhodou MVC je zásah obchodní logiky do UI a těžší implementace testů. MVC je starší model, a tedy jeho používání může být v kontextu
moderních uživatelských rozhraní obtížné.~\cite{guru99_mvc_vs_mvvm}

\paragraph{MVVM}
Vzor MVVM není vhodný pro projekty s jednoduchým UI vzhledem k jeho komplexnosti. Zároveň je náročné správně navrhnout ViewModel, což by 
mohlo vyústit k vysokému počtu bloků s duplicitním kódem.~\cite{gossman2006advantages}

\paragraph{Component-based}
Vzhledem k vytváření oddělených komponent, Component-based přidává do systému vyšší komplexitu. Každá její komponenta musí být přesně definována, 
vyvinuta a spravována, což může vyústit k složitým vzájemným závislostem. Pokud jsou zároveň různé komponenty vyvíjeny různými týmy či s rozdílnými technologiemi, může
jejich integrace představovat další překážku.~\cite{sakovich2025componentbased}

\subsection{Shrnutí a výběr architektury a architektonického vzoru}
Po shrnutí výhod a nevýhod všech zmíněných vzorů jsme získali lepší obrázek o tom, jak každý z nich funguje a k jakému účelu slouží. Nyní proto můžeme vybrat 
vzor nejvhodnější pro náš projekt. Z hlediska celkové architektury aplikace byla zvolena monolitická architektura, jelikož vyvíjený projekt svým rozsahem ani očekávaným zatížením nevyžaduje použití 
distribuovaných architektur. Veškerá aplikační logika, vizualizační vrstva i práce s daty jsou realizovány přímo na straně klienta. Monolitický přístup umožňuje jednodušší 
návrh, implementaci i nasazení aplikace a zároveň snižuje režii spojenou se správou komunikace mezi jednotlivými částmi systému. Pro účely této práce tak 
představuje přehledné a efektivní řešení, které odpovídá charakteru vyvíjené aplikace.

Náš projekt nedosahuje takových rozměrů, aby bylo vhodné pro něj využít vzor MVVM. Zároveň jeho obousměrný tok dat může v některých případech způsobit neočekávané vedlejší efekty. 
Proto je třeba hledat jednodušší řešení a vyvarovat se těm, které by mohly způsobit nadbytečné komplikace.

MVC je sice vhodný pro projekty se stejnými rozměry, jako má tato práce, avšak vhodnější volbou je Component-based, a to z několika následujících důvodů. MVC často vyžaduje explicitní instrukce
při změně modelu pro aktualizaci View, zároveň s růstem aplikace začíná být zajišťování závislostí a udržování oddělení částí mnohem náročnější. Component-based je založena na dynamičnosti
a její velkou výhodou, jak již bylo zmíněno v předchozí podsekci, je právě možnost rozšiřitelnosti s minimálním dopadem na celý systém.
Dle~\cite{reactnative_components} je React Native založený na komponentách a kromě základních a nativních komponent existují i další, vytvořené komunitou. Je proto vhodná pro náš projekt a můžeme 
říci, že pro zájmy této práce a v souladu se standardními postupy v softwarovém inženýrství byla vybrána správná architektura i architektonický vzor pro vyvíjenou aplikaci a můžeme pokračovat k poslední části návrhu.