\section{Testování uživatelského rozhraní}
Následující sekce se věnuje uživatelskému testování ověřující funkčnost UI/UX. 
Spočívá v interakci skutečných lidí s vytvořeným produktem a sledování jejich reakcí a chování. 
Testování je možné provádět několika způsoby včetně laboratorního testování pohybu očí. V každém případě je však 
nezbytným krokem pro zajištění příjemného a efektivního zážitku pro uživatele.~\cite{contentsquare_usability_testing}

\subsection{Výběr uživatelů}
Pro uživatelské testování byly vybrány 4 osoby ze sociálního okruhu autorky práce ve věku od 15 do 51 let.
Tři osoby využívají operační systém iOS na denní bázi, jedna z nich naopak disponuje zařízením se systémem Android. Testování však 
proběhlo ve všech případech na zařízení se systémem iOS. Vzhledem k jednotnému prostředí aplikace na obou platformách jsou však rozdíly 
při používání aplikace minoritní.
Jedna z osob je diagnostikována s poruchou barevného vidění, zbylé osoby bez potíží.

\subsection{Testovací scénáře}
Testování aplikace proběhlo na zařízeních iPhone XR a iPhone 11 Pro Max. Mobilní aplikace již byla předem nainstalovaná do zařízení, hlavním cílem 
testování bylo ověření funkčnosti a intuitivnosti uživatelského rozhraní a správné fungování rozšířené reality.

Jednotlivé testy proběhly pod vedení autorky práce, pokud byly potřebné prerekvizity, byly uživateli poskytnuty před samotným testováním. 
Po testování každý z testerů vypověděli zpětnou vazbu vztahující se k fungování aplikace pro následnou analýzu. Ukázky scénářů z uživatelského 
testování se nachází v příloze \ref{sec:příloha A}. Zahrnují detekování barvy za pomocí rozšířené reality, mazání barev, vytváření a ukládání fotografie 
i nalezení cesty k detailu barvy.

\subsection{Výsledky uživatelského testování}
Všichni uživatelé splnili jednotlivé testovací scénáře bez nápovědy či jiného zásahu. Uživatelé ocenili přehlednou navigaci, jedna z osob 
zdůraznila užitečnost zpoloprůhledněných tlačítek ve chvíli, kdy je není možné využívat.
Pro osobu s poruchou barevného vidění bylo vše zřetelné, při testování využila porovnání dvou záměnných barev, čímž aplikace prokázala funkčnost 
při snaze o pochopení rozdílnosti odstínů.
Ve funkčnosti aplikace uživatelé neidentifikovali zásadní problémy, v jednom případě proběhlo nepřesné umístění barvy v rozšířené realitě. 
Řešením tohoto problému bylo ponechání delší doby pro skenování okolí kamerou.


