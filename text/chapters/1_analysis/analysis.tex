\begin{chapterabstract}
První kapitola je věnována analýze, jenž je potřebná pro získání potřebných informací
a popsání požadavků softwarového produktu. Díky analýze se můžeme v následující kapitole
věnovat návrhu.

Následující stránky jsou věnovány analýze teorie barev a vnímání barev lidmi. Následně
je věnována pozornost známým barevným sadám a tvorbě barevných palet, extrakcí informací z fotografie
a využití AR k této extrakci. 

Prostor je věnován také existujícím aplikacím s podobnými funkcionalitami, díky kterým si upřesníme a definujeme
funkční a nefunkční požadavky. Na závěr specifikujeme požadavky pomocí modelu případů užití.
\end{chapterabstract}

\section{Teorie barev}
Teorie barev by se dala popsat jako soubor postupů a pravidel zahrnující 
používání primárních pigmentových barev, vytváření barevné harmonie, míchání barev
a jejich aplikaci.

Abychom se však mohli věnovat samotným teoriím a určit postupy, se kterými budeme k této práci
přistupovat, je třeba začít s úplnmými základy a definovat si, co vlastně je barva a jak je vnímána
lidským okem.


\subsection{Barva}
Vnímání barev je velmi subjektivní a již od dob Aristotela nám jejich zkoumání nepřineslo
jednotný názor, který by přesně definoval barvu.
Ten založil své poznatky na pozorování slunečního světla, které při odrazu či průchodu objektem
snižuje svou intenzitu nebo je ztmaveno. Vnímal tak barvu jako mísení, míchání, superpozici či juxtapozici
černé a bílé \cite{goethe1840}.

Tento názor se však brzy setkal s kritikou a byl nahrazen novými poznatky od dalších teoretiků.
% Newton, Young, Fresnel, Goethe - zkoumání barev (goethe1840)
% https://books.google.cz/books?hl=en&lr=&id=mujNEAAAQBAJ&oi=fnd&pg=PP13&dq=colour+definition&ots=ECMtqQBf1F&sig=-kCLAYmN6owutngTKMQAJ4Fu2fc&redir_esc=y#v=onepage&q&f=false
% https://d1wqtxts1xzle7.cloudfront.net/112207887/jaic_v32_01-libre.pdf?1709866938=&response-content-disposition=inline%3B+filename%3DColour_theory_Definition_fields_and_inte.pdf&Expires=1752252945&Signature=ZfOn9uc-IsbJP3TrC5MvZV4lSsAOte25rIBlngEMk7LidfCjxXVDtY9tmgN8RPonzWEm4AD0TRa-vkG0SGxMuVksHhNWawTge-7yJNuDxPnS5vOLtoCdWxS3Qpb1Uj1qdG~UnhJq8XXZFIubw9iDNmgUWR0fhSkD~ejasDRYdj-~ZjkHxcYEVLaazY~OFY3QhmTVcZ25B1bzPfnxpwZjDOMPBVPJX397~6qY~0ZsyctLy0jzs8ehTZahxoXmJz8vCDESfASXwe~4IDfXkmWs-2u1JjHVFVDsWb-HVVp~witzDU-VBO0odEnlIta0~WRoEeWiA6oyjz893LdCmRO8iw__&Key-Pair-Id=APKAJLOHF5GGSLRBV4ZA
% další research pro historii a vývoj názorů, odkud a co je barva 
% https://www.pantone.com/articles/color-fundamentals/what-is-color#:~:text=Color%20is%20defined%20as%20the,others%20are%20absorbed%20by%20it.
% pantone stránka - moderní definice barvy, dnešní chápání 

\subsubsection{Vlastnosti barev}

\subsection{Oko a barevné vidění}
% https://is.muni.cz/th/gpxge/Diplomova_prace.pdf
% struktura oka, tyčinky - černobílé, čípky - barevné vidění, proces předávání informace do mozku
% https://www.mendeley.com/catalogue/a9dda4be-621f-3d6f-b83d-343556b66070/
% trichromatické vidění - long(reddish), medium (greenish), and short (bluish) vawelenghts
% rozpoznávání barev v průběhu života
% https://dspace.cvut.cz/bitstream/handle/10467/91222/FBMI-BP-2020-Soukup-Jan-prace.pdf?sequence=-1&isAllowed=y
% teorie vnímání barev,  Young-Helmholtzova - tři typy čípků, Heringova teorie - proces
% spojení obou teorií - Dvoustupňová

\subsection{Teorie barev}

\subsection{Poruchy barevného vidění}
\section{Barevné sady a tvorba barevných palet}
\section{Extrakce informace z fotografie a AR}
\section{Rešerše existujících řešení}
\section{Požadavky}
\section{Model případu užití}

\subsection{Diagram případů užití}
% Modelové situace vizualizující funkční požadavky
% uživatel - Použití nápovědy, identifikace barvy, uložení fotografie, uložení a odstranění barvy, vyhledání barvy,
% umístění objektu do AR, získání barevných schémat

\subsection{Popis případů užití}

