\begin{chapterabstract}
První kapitola je věnována analýze, jenž je potřebná pro získání potřebných informací
a popsání požadavků softwarového produktu. Díky analýze se můžeme v následující kapitole
věnovat návrhu.

Následující stránky jsou věnovány analýze teorie barev a vnímání barev lidmi. Následně
je věnována pozornost známým barevným sadám a tvorbě barevných palet, extrakcí informací z fotografie
a využití AR k této extrakci. 

Prostor je věnován také existujícím aplikacím s podobnými funkcionalitami, díky kterým si upřesníme a definujeme
funkční a nefunkční požadavky. Na závěr specifikujeme požadavky pomocí modelu případů užití.
\end{chapterabstract}

\section{Teorie barev}
Teorie barev by se dala popsat jako soubor postupů a pravidel zahrnující 
používání primárních pigmentových barev, vytváření barevné harmonie, míchání barev
a jejich aplikaci.

Abychom se však mohli věnovat samotným teoriím a určit postupy, se kterými budeme k této práci
přistupovat, je třeba začít s úplnmými základy a definovat si, co vlastně je barva a jak je vnímána
lidským okem.


\subsection{Barva}
Vnímání barev je velmi subjektivní a již od dob Aristotela nám jejich zkoumání nepřineslo
jednotný názor, který by přesně definoval barvu.
Ten založil své poznatky na pozorování slunečního světla, které při odrazu či průchodu objektem
snižuje svou intenzitu nebo je ztmaveno. Vnímal tak barvu jako mísení, míchání, superpozici či juxtapozici
černé a bílé \cite{goethe1840}.

Tento názor se však brzy setkal s kritikou a byl nahrazen novými poznatky od dalších teoretiků.
% Newton, Young, Fresnel, Goethe - zkoumání barev (goethe1840)
% https://books.google.cz/books?hl=en&lr=&id=mujNEAAAQBAJ&oi=fnd&pg=PP13&dq=colour+definition&ots=ECMtqQBf1F&sig=-kCLAYmN6owutngTKMQAJ4Fu2fc&redir_esc=y#v=onepage&q&f=false
% https://d1wqtxts1xzle7.cloudfront.net/112207887/jaic_v32_01-libre.pdf?1709866938=&response-content-disposition=inline%3B+filename%3DColour_theory_Definition_fields_and_inte.pdf&Expires=1752252945&Signature=ZfOn9uc-IsbJP3TrC5MvZV4lSsAOte25rIBlngEMk7LidfCjxXVDtY9tmgN8RPonzWEm4AD0TRa-vkG0SGxMuVksHhNWawTge-7yJNuDxPnS5vOLtoCdWxS3Qpb1Uj1qdG~UnhJq8XXZFIubw9iDNmgUWR0fhSkD~ejasDRYdj-~ZjkHxcYEVLaazY~OFY3QhmTVcZ25B1bzPfnxpwZjDOMPBVPJX397~6qY~0ZsyctLy0jzs8ehTZahxoXmJz8vCDESfASXwe~4IDfXkmWs-2u1JjHVFVDsWb-HVVp~witzDU-VBO0odEnlIta0~WRoEeWiA6oyjz893LdCmRO8iw__&Key-Pair-Id=APKAJLOHF5GGSLRBV4ZA
% další research pro historii a vývoj názorů, odkud a co je barva 
% https://www.pantone.com/articles/color-fundamentals/what-is-color#:~:text=Color%20is%20defined%20as%20the,others%20are%20absorbed%20by%20it.
% pantone stránka - moderní definice barvy, dnešní chápání 

\subsubsection{Vlastnosti barev}

\subsection{Oko a barevné vidění}
% https://is.muni.cz/th/gpxge/Diplomova_prace.pdf
% struktura oka, tyčinky - černobílé, čípky - barevné vidění, proces předávání informace do mozku
% https://www.mendeley.com/catalogue/a9dda4be-621f-3d6f-b83d-343556b66070/
% trichromatické vidění - long(reddish), medium (greenish), and short (bluish) vawelenghts
% rozpoznávání barev v průběhu života
% https://dspace.cvut.cz/bitstream/handle/10467/91222/FBMI-BP-2020-Soukup-Jan-prace.pdf?sequence=-1&isAllowed=y
% teorie vnímání barev,  Young-Helmholtzova - tři typy čípků, Heringova teorie - proces
% spojení obou teorií - Dvoustupňová

\subsection{Teorie barev}

\subsection{Poruchy barevného vidění}
\section{Barevné sady a tvorba barevných palet}
%https://dspace.cuni.cz/bitstream/handle/20.500.11956/188964/130380558.pdf?sequence=1
% krátká zmínka o historických paletách - jen o jejich existenci a posunutí oproti těm od goetheho a Newtona
% barevne system, reprezentace (aditivni apod) RGB, CMYK - https://naos-be.zcu.cz/server/api/core/bitstreams/830d5224-cbd9-42c3-af0d-3ff9dd034966/content
% https://www.premocz.eu/barvy-rgb-cmyk-a-pantone
% https://www.shutterstock.com/blog/color-scheme-definitions-types-examples
\section{Extrakce informace z fotografie a AR}
% https://scholar.google.com/scholar_lookup?journal=PLoS%20One&title=Accurate%20device-independent%20colorimetric%20measurements%20using%20smartphones.&author=M%20Nixon&author=F%20Outlaw&author=TS%20Leung&volume=15&issue=3&publication_year=2020&pages=e0230561&pmid=32214340&doi=10.1371/journal.pone.0230561&
% pro focení z kamery pro nejvyšší věrohodnost
% https://pmc.ncbi.nlm.nih.gov/articles/PMC10270580/?utm_source=chatgpt.com 
% extrakce barev z kamery, barevná korekce
% file:///C:/Users/sarka/Downloads/13254-Article%20Text-15623-1-10-20231226%20(2).pdf 
% aplikace pro barvoslepé - popis dalších možných řešení
\section{Rešerše existujících řešení}
% file:///C:/Users/sarka/Downloads/13254-Article%20Text-15623-1-10-20231226%20(2).pdf%
% jedna z aplikací pomáhající barvoslepému člověku rozpoznávat barvy
% Color Blind Pal, Color identifier, Paleto, Color Name, Colore name Re... viz telefon
\section{Požadavky}
Po zhodnocení již existujících řešení, analýze a pro splnění další dílčí části zadání této práce je třeba zadefinovat si požadavky
pro aplikaci, jenž je hlavním cílem této práce. Jejich specifikování je podstatné pro další fáze vývoje,
zjištění důležitých prvků a funkcionalit a stanovení priorit, kterými se budeme ve vývoji řídit.

Každý požadavek vyžaduje uvedení informací umožňující nám je specifikovat pro lepší pochopení. V této práci
využíváme následující údaje dle~\cite{analyza_a_sber}. % https://moodle-vyuka.cvut.cz/pluginfile.php/898714/mod_resource/content/11/03.prednaska.pdf

\begin{itemize}
  \item \textbf{Název} uvádějící stručný popis požadavku.
  \item \textbf{Identifikátor} umožňující snazší odkazování na požadavek.
  \item \textbf{Popis} upřesňující detaily požadavku. Jedná se o nejdůležitější část.
  \item \textbf{Priorita} identifikující důležitost požadavku při vývoji. Specifikuje nám, 
  bez kterých požadavků by aplikace nemohla existovat a které lze zrealizovat, ale nejsou pro tento běh významné. 
  Pro rozdělení priorit požadavků využijeme MoSCoW metodu rozdělující prioritu do čtyř kategorií~\cite{moscow}:
   \begin{enumerate}
    \item \textbf{Must have} definuje požadavky, které je nutné zahrnout ve finálním produktu.
    \item \textbf{Should have} popisuje význačné požadavky, které by měl produkt obsahovat, ale aplikace bez nich může fungovat.
    \item \textbf{Could have} požadavky jsou žádoucí, ale pouze v případě, že nevyžadují příliš velké úsilí či náklady.
    \item \textbf{Will not have} jsou požadavky, které je v zájmu zainteresovaných stran implementovat, ale ne v aktuální časové verzi.
    \end{enumerate}
\end{itemize}

Požadavky se dělí na \textbf{funkční} a \textbf{nefunkční} požadavky. \textbf{Funkční požadavky} jsou takové požadavky, které
definují chování systému. \textbf{Nefunkční požadavky} určují omezení informačního systému a mají dopad na zvolení architektury a dodržování standardů.
Níže jsou vypsány veškeré požadavyk pro tuto práci i s popisem a určením priorit.

\subsection{Funkční požadavky}
\paragraph{FP1 – Identifikace barev} Aplikace rozpozná barvy podle kamery. \emph{(Must have)} 
% FP1 - Identifikace barev 
% Aplikace je schopná identifikovat barvy na základě kamery
% Priorita: Must Have
\paragraph{FP2 – Tvorba barevných schémat} Na základě vybrané barvy se automaticky generují korespondující barevné palety. \emph{(Must have)} 
% FP2 - Tvorba barevných schémat
% Aplikace vytváří na základě zvolené barvy další korespondující barevné palety
% Priorita: Must have
\paragraph{FP3 – Ukládání a vyhledávání barev} V aplikaci je možné ukládat identifikované barvy a vyhledávat již uložené v knihovně barev. \emph{(Must have)} 
% FP3 - Ukládání a vyhledávání barev
% V aplikaci je možné ukládat identifikované barvy a vyhledávat již uložené v knihovně barev.
% Priorita: Must have
\paragraph{FP4 – Odstranění barev z knihovny} Možnost mazat jednotlivé či více uložených barev. \emph{(Should have)} 
% FP4 - Odstranění barev z knihovny
% Lze odstranit uloženou barvu či více barev z knihovny
% Priorita: Should have
\paragraph{FP5 – Nápověda} Asistent pomáhající k lepší orientaci v aplikace. \emph{(Should have)} 
% FP5 - Nápověda
% Aplikace disponuje asistentem pro orientaci v aplikaci
% Priorita: Should have
\paragraph{FP6 – Vizualizace identifikované barvy} Extrahovaná barva je zobrazena na bodě detekce v kameře. \emph{(Must have)} 
% FP6 - Vizualizace identifikované barvy
% Extrahovaná barva je zobrazená v aplikaci
% Priorita: Must have
\paragraph{FP7 – Interakce v rozšířené realitě} Vytvoření virtuálního bodu za pomocí rozšířené reality. \emph{(Must have)} 
% FP7 - Interakce v rozšířené realitě
% Vytvoření virtuálního bodu za pomocí rozěířené reality
% Priorita: Must have
\paragraph{FP8 – Rozšířená interakce v AR} Vytvoření více virtuálních bodů v AR pro porovnání barev. \emph{(Could have)} 
% FP8 - Rozšířená interkace v AR
% Lze vytvořit více virtuálních bodů za pomocí AR pro porovnání barev
% Priorita: Could have
\paragraph{FP9 – Extrakce z fotografie} Extrakce barvy z fotografie. \emph{(Could have)} 
% FP9 - Extrakce z fotografie
% Extrakce barvy z fotografie
% Priorita: Could have
\paragraph{FP10 – Uložení vytvořeného snímku} Snímek z kamery lze uložit do zařízení. \emph{(Could have)} 
% FP10 - Uložení fotografie do galerie
% Vytvořenou fotografii z kamery lze uložit do galerie
% Priorita: Could have
\paragraph{FP11 – Míchání uložených barev} Aplikace vytváří další barvy z již uložených. \emph{(Will not have)} 
% FP11 - Blending uložených barev
% Aplikace může vytvářet další barvy z již uložených
% Priorita: Will not have
\paragraph{FP12 – Tvorba dalších barevných schémat} Aplikace vytváří další typy barevných palet. \emph{(Will not have)} 
% FP12 - Tvorba více barevných schémat
% Aplikace umí vytvářet více barevných palet
% Priorita: Will not have

\subsection{Nefunkční požadavky}
\paragraph{NP1 – Multiplatformní aplikace} Aplikace je vyvíjena pro iOS i Android platformy. \emph{(Must have)} 
% NP1 - Multiplatformní aplikace
% Aplikace je vyvíjena pro iOS i Android platformy.
% Priorita: Must have
\paragraph{NP2 – Použití rozšířené reality} Využívají se komponenty z AR - interakce se světem, umístění objektů. \emph{(Must have)} 
% NP2 - Použití AR
% Využívané komponenty z AR - interakce se světem, umístění objektů
% Priorita: Must have
\paragraph{NP3 – Přístupnost} Aplikace má jednoduché UI a je přístupná všeobecnému publiku včetně osob s poruchami barevného vidění. \emph{(Must have)} 
% NP3 - Přístupnost
% Aplikace má jednoduché UI a je přístupná všeobecnému publiku včetně osob s poruchami barevného vidění.
% Priorita: Must have
\paragraph{NP4 – Rozšiřitelnost} Aplikaci lze dále jednoduše rozšiřovat v rámci tvorby dalších palet či větší interakce s extrahovanými barvami. \emph{(Should have)} 
% NP4 - Rozšiřitelnost
% Aplikaci lze dále jednoduše rozšiřovat v rámci tvorby dalších palet či větší interakce s extrahovanými barvami.
% Priorita: Should have
\paragraph{NP5 – Výkon} Aplikace lokálně funguje plynule bez sekání a prodlev, detekce je rychlá bez významného zaznamenání při používání. \emph{(Must have)} 
% NP5 - Výkon
% Aplikace lokálně funguje plynule bez sekání a prodlev, detekce je rychlá bez významného zaznamenání při používání.
% Priorita: Must have
\paragraph{NP6 – Technická dostupnost} Aplikace je dostupná pro modely Apple iPhone 8 a vyšší a Android zařízení se systémem Android 7.0 a vyšší (s ARCore podporou). \emph{(Should have)} 
% NP6 - Technická dostupnost
% Aplikace je dostupná pro modely Apple iPhone 8 a vyšší a Android zařízení se systémem Android 7.0 a vyšší (s ARCore podporou)
% Priorita: Should have
\section{Model případů užití}
Model případů užití prezentuje využití funkčních požadavků na konkrétních případech, čímž pomáhá
vysvětlit požadované funkcionality a poskytuje pro ně detailnější popis, tedy případy užití. Model případů užití je tvořen
seznamem aktérů a diagramem případů užití, které jsou níže popsány.~\cite{analyza_a_sber}

\subsection{Seznam případů užití}


\paragraph{UC1. Detekce barvy z kamery}\mbox{}\\
Po rozkliknutí "Camera" v hlavním menu se uživateli zobrazí kamera a v horní části nejprve dojde k inicializaci.
Ve chvíli, kdy je kamera připravena, lze kliknout na požadovaný objekt, který vygeneruje objekt na místě stisku spolu s informačním
boxem s názvem barvy, ukázkou a přesným kódem. Pro detekci jiné barvy stačí opět kliknout na místo, kde má barva být detekována.

\paragraph{UC2. Detekce barvy z obrázku}\mbox{}\\
Po rozkliknutí "Camera" v hlavním menu je možné v levém spodním rohu kliknout na ikonu galerie. Po povolení přístupu může uživatel
vybrat požadovanou fotografii či obrázek ze své galerie, který je načten na obrazovku. Uživatel může libovolně vybírat stiskem místo pro
určení barvy, kde se následně vytvoří orámované kolečko. Vybraná barva spolu se jménem a kódem se objeví na informačním boxu uprostřed obrazovky.

\paragraph{UC3. Zobrazení detailu barvy}\mbox{}\\
Pro zobrazení detailu barvy je možné zvolit dva postupy. První po výběru a detekci barvy z kamery či obrázku, a to kliknutím
na obrázek palety ve spodní části aplikace. Druhým způsobem je výběr "Library" v hlavním menu a výběrem požadované barvy. Na stránce
s detailem je zobrazena barva, její název, HEX, CMYK a HSV kód spolu s monochromatickou, analogickou a komplementární paletou.

\paragraph{UC4. Uložení barvy}\mbox{}\\
Po výběru barvy z kamery či obrázku je možné uložit zvolenou barvu do knihovny pomocí stisku ikony uložení vedle ikony galerie. Barva se následně
objeví v sekci "Library", ke které má uživatel neomezený přístup.

\paragraph{UC5. Prohlížení uložených barev}\mbox{}\\
Uživatel může v sekci "Library" v hlavním menu získat přístup k celé knihovně barev uložených z kamery či obrázku. Pomocí posuvníku lze procházet barvy
seřazené od posledního uložení po nejstarší.

\paragraph{UC6. Smazání uložených barev}\mbox{}\\
V sekci "Library" v hlavním menu získá uživatel přístup ke knihovně barev. Pro výběr barvy lze podržet stisk na požadovaném políčku, přičemž se ve vrchní části
objeví tlačítka pro vymazání barvy či zrušení výběru. Uživatel takto může vybírat a mazat více barev najednou. Pro výběr více než jedné barvy stačí krátký stisk,
při opětovném stisku je barva z výběru odebrána.

\paragraph{UC7. Zobrazení nápovědy}\mbox{}\\
Nápověda použití aplikace je uživateli zobrazena při prvním spuštění aplikace. V případě, že je nápověda vyžadována v průběhu používání,
lze ji najít v sekci "Help" v hlavním menu. Nápověda obsahuje plnohodnotný návod na používání aplikace včetně vizuálního doprovodu.


\subsection{Diagram případů užití}
\begin{figure}[!ht]
    \centering
    \includegraphics[width=1\linewidth]{images/diagram.png}
    \caption{Diagram případů užití dle~\cite{analyza_a_sber}.}
    \label{fig:Diagram případů užití}
\end{figure}
\section{Stavový diagram}
bubu