\section{Rešerše existujících řešení}
% file:///C:/Users/sarka/Downloads/13254-Article%20Text-15623-1-10-20231226%20(2).pdf%
% jedna z aplikací pomáhající barvoslepému člověku rozpoznávat barvy
% Color Blind Pal, Color identifier, Paleto, Color Name, Colore name Re... viz telefon

Analýza a rešerše již existujících aplikací detekující barvu z okolního prostředí a vytvářející
odpovídající barevné palety je jedním z klíčových bodů pro popsání silných a slabých stránek, které
lze využít pro definování požadavků pro výslednou aplikaci.
Výsledek z této rešerše bude využit k inspiraci z kladných aspektů využitých v konkurenčních aplikacích a
k eliminaci nalezených nedostatků.
V následujících podsekcích budou představeny čtyři vybrané aplikace s podobnými funkcionalitami, jejich popis včetně
diskutování slabých a silných stránek. Vyhledávání aplikací probíhalo na oficiální distribuční platformě AppStore s pomocí klíčových slov 
zaměřených na detekci barev s využitím i bez využití rozšířené reality a práci s paletami. Do výběru byly zařazeny aplikace, které splňovaly alespoň jednu z klíčových funkcionalit pro navrhovanou 
aplikaci. Veškeré níže zmíněné aplikace jsou dostupné v bezplatné verzi s možností premium placené verze. 

\subsection{Color Blind Pal}
Aplikace Color Blind Pal je rozšířenou aplikací umožňující osobám trpícím barvoslabostí rozpoznávat okolní barvy, zároveň
však umožňuje osobám s normálním viděním zobrazit svět tak, jak jej vnímají pacienti s různými diagnostikami.
Pro úpravu světa jsou využívány různé filtry, které se snaží přiblížit co nejvíce uvedené vadě v barevném vidění. Jedná se spíše o
velmi přibližnou ilustraci, vzhledem k tomu, že každá vada je u každého velmi specifická. 
Samotné ovládání je velmi jednoduché a intuitivní, aplikace pracuje pouze s kamerou a barevným spektrem ve spodní části. Nevýhodou může být neustále přeskakující
detektor barvy, jelikož se kamera neustále snaží zaostřovat a adaptovat. Při detekci je zobrazen také název barvy, v případě jemných odstínů je však
tato detekce omezená, což může být v některých případech klíčové. Neexistuje taktéž žádná možnost uložení barev.

\begin{figure}[!ht]
    \centering
    \includegraphics[width=0.6\linewidth]{images/colorblindpal.png}
    \caption{Ukázky aplikace Color Blind Pal.~\cite{color_blind_pal_app}}
    \label{fig:ColorBlindPal}
\end{figure}

\subsection{Color Identifier: Color Picker}
Aplikace Color Identifier: Color Picker se věnuje rozpoznávání barev z kamery a tvorbě vlastních barevných palet. Umožňuje
nejen ukládat detekované barvy, ale také z barevného spektra vybírat další odstíny, které si lze uložit do knihovny.
Aplikace dále nabízí také konvertor barev dle HEX, RGB či CMYK kódu nebo pomocí výběru ze spektra. V nové verzi je uživatelům nabízen také
režim pro barvoslepé nabízející zjednodušené barevné palety.
Aplikace je velmi uživatelsky přívětivá, disponuje podporou více jazyků a přesně určuje detekované barvy. Ty jsou zobrazeny v boxu spolu
s dalšími kódy. Po výběru barvy v knihovně lze vygenerovat 5 různých barevných palet v počtu až 5 barev, které si lze také uložit.
Ačkoliv je aplikace velmi přehledná, obsahuje invazivní množství reklam a jen velmi omezené možnosti v bezplatné verzi, některé z nich jsou pro běžné využití nadbytečné (Např. Generování pouze 1 barvy v paletě).
Zároveň nelze v režimu výběru ve fotoaparátu zvolenou barvu při pohybu kamery zachovat.

\begin{figure}[!ht]
    \centering
    \includegraphics[width=0.7\linewidth]{images/identifier.png}
    \caption{Ukázky aplikace Color Identifier.~\cite{color_identifier_color_picker_app}}
    \label{fig:ColorIdentifier}
\end{figure}

\subsection{Paleto}
Paleto se orientuje primárně na tvorbu palet a míchání barev z konkrétních odstínů včetně volby poměru. Poskytuje možnost generování palety z fotografie či obrázku, hledání
barvy v obáshlé knihovně i ukládání vytvořených odstínů.
V této aplikaci nelze detekovat konkrétní barvu, generování z fotoaparátu poskytuje pouze omezený počet nalezených barev. Aplikace je jinak minimalistická a neobsahuje nadbytečné
funkcionality, které by komplikovaly její použití.

\begin{figure}[!htpb]
    \centering
    \includegraphics[width=0.7\linewidth]{images/paleto.png}
    \includegraphics[width=0.7\linewidth]{images/color_name_recognizer.png}
    \caption{Ukázky aplikace Paleto\cite{paleto_mixing_colors_app} a Color Name Recognizer\cite{color_name_recognizer_camera_app}.}
    \label{fig:Paleto}
\end{figure}

\subsection{Color Name Recognizer}
Mobilní aplikace Color Name Recognizer se také věnuje detekci barev z kamery. Nabízí box s názvem, ukázkou zvolené barvy a kódy RGB, CIE a HSV.
Nástroj obsahuje voličem velikosti detekované plochy, přiblížením, možností zobrazení 1~až 3 podobných barev z palety a lupou pro konkrétnější výběr oblasti. Výhodou je také možnost
pozastavení kamery pro získání detekované barvy. Detektor je ovšem neustále zapnutý, což má za následek stejnou problematiku, se kterou se potýká
aplikace Color Blind Pal. Box neustále mění detekované informace a mění se také generované barvy na paletě. Dochází tak k přeskokům, které pro uživatele nemusí působit přívětivě.

\subsection{Shrnutí}
Uvedené aplikace se potýkají s podobnými nedostatky i výhodami, zároveň ukazují, jak důležitá je volba vizuálních prvků. Uvedená rešerše poslouží jako podklad k analýze požadavků, kterým se
věnuje následující sekce.

\clearpage
