\section{Model případů užití}
Model případů užití prezentuje využití funkčních požadavků na konkrétních případech, čímž pomáhá
vysvětlit požadované funkcionality a poskytuje pro ně detailnější popis, tedy případy užití. Model případů užití je tvořen
seznamem aktérů a diagramem případů užití, které jsou níže popsány.~\cite{analyza_a_sber}

\subsection{Seznam případů užití}


\paragraph{UC1. Detekce barvy z kamery}\mbox{}\\
Po rozkliknutí "Camera" v hlavním menu se uživateli zobrazí kamera a v horní části nejprve dojde k inicializaci.
Ve chvíli, kdy je kamera připravena, lze kliknout na požadovaný objekt, který vygeneruje objekt na místě stisku spolu s informačním
boxem s názvem barvy, ukázkou a přesným kódem. Pro detekci jiné barvy stačí opět kliknout na místo, kde má barva být detekována.

\paragraph{UC2. Detekce barvy z obrázku}\mbox{}\\
Po rozkliknutí "Camera" v hlavním menu je možné v levém spodním rohu kliknout na ikonu galerie. Po povolení přístupu může uživatel
vybrat požadovanou fotografii či obrázek ze své galerie, který je načten na obrazovku. Uživatel může libovolně vybírat stiskem místo pro
určení barvy, kde se následně vytvoří orámované kolečko. Vybraná barva spolu se jménem a kódem se objeví na informačním boxu uprostřed obrazovky.

\paragraph{UC3. Zobrazení detailu barvy}\mbox{}\\
Pro zobrazení detailu barvy je možné zvolit dva postupy. První po výběru a detekci barvy z kamery či obrázku, a to kliknutím
na obrázek palety ve spodní části aplikace. Druhým způsobem je výběr "Library" v hlavním menu a výběrem požadované barvy. Na stránce
s detailem je zobrazena barva, její název, HEX, CMYK a HSV kód spolu s monochromatickou, analogickou a komplementární paletou.

\paragraph{UC4. Uložení barvy}\mbox{}\\
Po výběru barvy z kamery či obrázku je možné uložit zvolenou barvu do knihovny pomocí stisku ikony uložení vedle ikony galerie. Barva se následně
objeví v sekci "Library", ke které má uživatel neomezený přístup.

\paragraph{UC5. Prohlížení uložených barev}\mbox{}\\
Uživatel může v sekci "Library" v hlavním menu získat přístup k celé knihovně barev uložených z kamery či obrázku. Pomocí posuvníku lze procházet barvy
seřazené od posledního uložení po nejstarší.

\paragraph{UC6. Smazání uložených barev}\mbox{}\\
V sekci "Library" v hlavním menu získá uživatel přístup ke knihovně barev. Pro výběr barvy lze podržet stisk na požadovaném políčku, přičemž se ve vrchní části
objeví tlačítka pro vymazání barvy či zrušení výběru. Uživatel takto může vybírat a mazat více barev najednou. Pro výběr více než jedné barvy stačí krátký stisk,
při opětovném stisku je barva z výběru odebrána.

\paragraph{UC7. Zobrazení nápovědy}\mbox{}\\
Nápověda použití aplikace je uživateli zobrazena při prvním spuštění aplikace. V případě, že je nápověda vyžadována v průběhu používání,
lze ji najít v sekci "Help" v hlavním menu. Nápověda obsahuje plnohodnotný návod na používání aplikace včetně vizuálního doprovodu.


\subsection{Diagram případů užití}
\begin{figure}[!ht]
    \centering
    \includegraphics[width=1\linewidth]{images/diagram.png}
    \caption{Diagram případů užití dle~\cite{analyza_a_sber}.}
    \label{fig:Diagram případů užití}
\end{figure}