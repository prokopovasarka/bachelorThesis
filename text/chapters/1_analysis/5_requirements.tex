\section{Požadavky}
Po zhodnocení již existujících řešení a analýze samotné je třeba zadefinovat si požadavky
pro aplikaci, jenž je výsledkem této práce. Jejich specifikování je podstatné pro další fáze vývoje,
zjištění důležitých prvků a funkcionalit a stanovení priorit, kterými se budeme ve vývoji řídit.

Každý požadavek vyžaduje uvedení informací umožňující nám je specifikovat pro lepší pochopení. V této práci
využíváme následující údaje dle. % https://moodle-vyuka.cvut.cz/pluginfile.php/898714/mod_resource/content/11/03.prednaska.pdf

\begin{itemize}
  \item \textbf{Název} uvádějící stručný popis požadavku.
  \item \textbf{Identifikátor} umožňující snazší odkazování na požadavek.
  \item \textbf{Popis} upřesňující detaily požadavku. Jedná se o nejdůležitější část.
  \item \textbf{Priorita} identifikující důležitost požadavku při vývoji. Specifikuje nám, 
  bez kterých požadavků by aplikace nemohla existovat a které lze zrealizovat, ale nejsou pro tento běh významné. 
  Pro rozdělení priorit požadavků využijeme MoSCoW metodu rozdělující prioritu do čtyř kategorií~\cite{moscow}:
   \begin{enumerate}
    \item \textbf{Must have} definuje požadavky, které je nutné zahrnout ve finálním produktu.
    \item \textbf{Should have} popisuje význačné požadavky, které by měl produkt obsahovat, ale aplikace bez nich může fungovat.
    \item \textbf{Could have} požadavky jsou žádoucí, ale pouze v případě, že nevyžadují příliš velké úsilí či náklady.
    \item \textbf{Will not have} jsou požadavky, které je v zájmu zainteresovaných stran implementovat, ale ne v aktuální časovém verzi.
    \end{enumerate}
\end{itemize}

Pžadavky se dělí na \textbf{funkční} a \textbf{nefunkční} požadavky. \textbf{Funkční požadavky} jsou takové požadavky, které
definují chování systému. \textbf{Nefunkční požadavky} určují omezení informačního systému a mají dopad na zvolení architektury a dodržování standardů.
Níže jsou vypsány veškeré požadavyk pro tuto práci i s popisem a určením priorit.

\subsection{Funkční požadavky}
\paragraph{FP1 - Identifikace barev} Aplikace rozpozná barvy podle kamery. \emph{(Must have)} 
% FP1 - Identifikace barev 
% Aplikace je schopná identifikovat barvy na základě kamery
% Priorita: Must Have
\paragraph{FP2 - Tvorba barevných schémat} Na základě vybrané barvy se automaticky generují korespondující barevné palety. \emph{(Must have)} 
% FP2 - Tvorba barevných schémat
% Aplikace vytváří na základě zvolené barvy další korespondující barevné palety
% Priorita: Must have
\paragraph{FP3 - Ukládání a vyhledávání barev} V aplikaci je možné ukládat identifikované barvy a vyhledávat již uložené v knihovně barev. \emph{(Must have)} 
% FP3 - Ukládání a vyhledávání barev
% V aplikaci je možné ukládat identifikované barvy a vyhledávat již uložené v knihovně barev.
% Priorita: Must have
\paragraph{FP4 - Odstranění barev z knihovny} Možnost mazat jednotlivé či více uložených barev. \emph{(Should have)} 
% FP4 - Odstranění barev z knihovny
% Lze odstranit uloženou barvu či více barev z knihovny
% Priorita: Should have
\paragraph{FP5 - Nápověda} Asistent pomáhající k lepší orientaci v aplikace. \emph{(Should have)} 
% FP5 - Nápověda
% Aplikace disponuje asistentem pro orientaci v aplikaci
% Priorita: Should have
\paragraph{FP6 - Vizualizace identifikované barvy} Extrahovaná barva je zobrazena na bodě detekce v kameře. \emph{(Must have)} 
% FP6 - Vizualizace identifikované barvy
% Extrahovaná barva je zobrazená v aplikaci
% Priorita: Must have
\paragraph{FP7 - Interakce v rozšířené realitě} Vytvoření virtuálního bodu za pomocí rozšířené reality. \emph{(Must have)} 
% FP7 - Interakce v rozšířené realitě
% Vytvoření virtuálního bodu za pomocí rozěířené reality
% Priorita: Must have
\paragraph{FP8 - Rozšířená interakce v AR} Vytvoření více virtuálních bodů v AR pro porovnání barev. \emph{(Could have)} 
% FP8 - Rozšířená interkace v AR
% Lze vytvořit více virtuálních bodů za pomocí AR pro porovnání barev
% Priorita: Could have
\paragraph{FP9 - Extrakce z fotografie} Extrakce barvy z fotografie. \emph{(Could have)} 
% FP9 - Extrakce z fotografie
% Extrakce barvy z fotografie
% Priorita: Could have
\paragraph{FP10 - Uložení vytvořeného snímku} Snímek z kamery lze uložit do zařízení. \emph{(Could have)} 
% FP10 - Uložení fotografie do galerie
% Vytvořenou fotografii z kamery lze uložit do galerie
% Priorita: Could have
\paragraph{FP11 - Míchání uložených barev} Aplikace vytváří další barvy z již uložených. \emph{(Will not have)} 
% FP11 - Blending uložených barev
% Aplikace může vytvářet další barvy z již uložených
% Priorita: Will not have
\paragraph{FP12 - Tvorba dalších barevných schémat} Aplikace vytváří další typy barevných palet. \emph{(Will not have)} 
% FP12 - Tvorba více barevných schémat
% Aplikace umí vytvářet více barevných palet
% Priorita: Will not have

\subsection{Nefunkční požadavky}
\paragraph{NP1 - Multiplatformní aplikace} Aplikace je vyvíjena pro iOS i Android platformy. \emph{(Must have)} 
% NP1 - Multiplatformní aplikace
% Aplikace je vyvíjena pro iOS i Android platformy.
% Priorita: Must have
\paragraph{NP2 - Použití rozěířené reality} Využívají se komponenty z AR - interakce se světem, umístění objektů. \emph{(Must have)} 
% NP2 - Použití AR
% Využívané komponenty z AR - interakce se světem, umístění objektů
% Priorita: Must have
\paragraph{NP3 - Přístupnost} Aplikace má jednoduché UI a je přístupná všeobecnému publiku včetně osob s poruchami barevného vidění. \emph{(Must have)} 
% NP3 - Přístupnost
% Aplikace má jednoduché UI a je přístupná všeobecnému publiku včetně osob s poruchami barevného vidění.
% Priorita: Must have
\paragraph{NP4 - Rozšiřitelnost} Aplikaci lze dále jednoduše rozšiřovat v rámci tvorby dalších palet či větší interakce s extrahovanými barvami. \emph{(Should have)} 
% NP4 - Rozšiřitelnost
% Aplikaci lze dále jednoduše rozšiřovat v rámci tvorby dalších palet či větší interakce s extrahovanými barvami.
% Priorita: Should have
\paragraph{NP5 - Výkon} Aplikace lokálně funguje plynule bez sekání a prodlev, detekce je rychlá bez významného zaznamenání při používání. \emph{(Must have)} 
% NP5 - Výkon
% Aplikace lokálně funguje plynule bez sekání a prodlev, detekce je rychlá bez významného zaznamenání při používání.
% Priorita: Must have
\paragraph{NP6 - Technická dostupnost} Aplikace je dostupná pro modely Apple iPhone 8 a vyšší a Android zařízení se systémem Android 7.0 a vyšší (s ARCore podporou). \emph{(Should have)} 
% NP6 - Technická dostupnost
% Aplikace je dostupná pro modely Apple iPhone 8 a vyšší a Android zařízení se systémem Android 7.0 a vyšší (s ARCore podporou)
% Priorita: Should have