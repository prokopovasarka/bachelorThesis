\section{Rozšířená realita a její využití}
Rozšířená realita ("AR") je definována jako integrace digitálních informací do prostředí uživatele
v reálném čase. Technologie rozšířené reality překrývá obsah reálného světa a obohacuje jeho vnímání bez
nahrazování reality samotné.~\cite{ibm_augmented_reality}

\subsection{Virtuální realita vs rozšířená realita}
Virtuální realita i rozšířená realita jsou v digitálním světě již poměrně známé a jejich využití
je na vzestupu. Oba procesy však fungují velmi odlišně, proto je třeba ujasnit si rozdíly mezi těmito 
výrazy a konkrétněji definovat jejich využití.

Systémy rozšířené reality, jak je již definování výše, začleňují digitální informace do reálného světa, ale samotné
vnímání reality nijak nemění. Naopak virtuální realita ("VR") je popisována jako trojrozměrný svět generovaný počítačem.
Simuluje tedy prostředí vnímané uživazelem a umožňuje s ním interagovat, a tedy simulovat jeden či více smyslů. Fúze obou 
principů se nazývá "mixed reality".~\cite{gaol_frontiers_ar_mr_education_2022}

\subsection{Princip fungování}
Zařízení využívající rozšířenou realitu nejprve přijímá videosignál ze zorného pole uživatele (např. kamery) a snímá
okolní prostředí a fyzické objekty v tomto poli. Může to taktéž zahrnovat sběr dat z GPS, laserů či gyroskopů.

Software rozšířené reality zpracovává a skenuje toto přijímané prostředí. Snaží se identifikovat objekty a vlastnosti prostředí, které
by bylo možné rozšířit. Tato identifikace může potenciálně využívat umělou inteligenci k rozpoznání objektů či různé senzory.

Ze softwaru pokračují informace zpět do zařízení, kde je generovaný obsah překrýván do zorného pole uživatele, a to ve správné perspektivě
a orientaci. Pro interakci je možné využívat příkazy prostřednictvím fyzických gest, hlasu či dotykem obrazovky.

\subsection{Možnosti využití}
Dnes existuje mnoho možností, jak využívat rozšířenou realitu. V případě vzdělávání je možné zpřístupnit studentům lepší porozumění látky
pomocí různých 3D modelů a simulací. To může například zahrnovat prozkoumávání historických budov, či dokonce celých, dnes již neexistujících měst.

Její využití však lze objevit i v herním průmyslu, kde se herní charaktery a postavy objevují v reálném světě. Příkladem může být známá aplikace
Pokemon GO, která umožňuje uživatelům dosáhnout ještě větší interakce se světem, čímž pomáhá ještě intenzivnějšímu prožitku ze hry.

Ve zdravotnictví existují AR systémy pro trénink, plánování operací pro preciznější výsledky a vzdělávání pacientů. Rozšířená realita tedy rozhodně neslouží
pouze pro zábavní průmysl, může sloužit také jako podpora v oblastech, kde by vytvoření fyzického objektu znamenalo nereálné výdaje či čas na realizaci.

\subsection{Druhy rozšířené reality}
Rozlišujeme dva základní typy rozšířené reality. Rozlišují se dle druhů spouštěčů na "marker-based" a "marker-less".

\paragraph{Marker-based AR}\mbox{}\\
"Marker-based AR" je realita založená na značkách, kterými jsou v tomto případě fyzické objekty vytvářející nějaký fyzický spouštěč. Může jím být
QR kód, obrázek či jiná, předem definovaná značka. Při její detekci se spustí požadovaná akce rozšířené reality. Tento typ je dostupný bez omezení a je
velmi flexibilní. Zároveň je méně nákladná.

\paragraph{Marker-less AR}\mbox{}\\
"Marker-less AR" nevyžaduje narozdíl od předchozího typu žádný konkrétní spouštěč. Spoléhá na určené senzory, jakými může být GPS, akcelerometr či kamera, mapuje prostředí
v reálném čase a snaží se mu porozumět. Pomocí různých algoritmů a strojového vidění AR sama určuje umístění digitálního obsahu, zajišťující tím dynamičtější a spontánnější zážitek.~\cite{ibm_augmented_reality}