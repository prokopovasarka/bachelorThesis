\section{Teorie barev}
Teorie barev by se dala popsat jako soubor postupů a pravidel zahrnující 
používání primárních pigmentových barev, vytváření barevné harmonie, míchání barev
a jejich aplikaci.

Abychom se však mohli věnovat samotným teoriím a určit postupy, se kterými budeme k této práci
přistupovat, je třeba začít s úplnmými základy a definovat si, co vlastně je barva a jak je vnímána
lidským okem.


\subsection{Barva}
Vnímání barev je velmi subjektivní a již od dob Aristotela nám jejich zkoumání nepřineslo
jednotný názor, který by přesně definoval barvu.
Ten založil své poznatky na pozorování slunečního světla, které při odrazu či průchodu objektem
snižuje svou intenzitu nebo je ztmaveno. Vnímal tak barvu jako mísení, míchání, superpozici či juxtapozici
černé a bílé \cite{goethe1840}.

Tento názor se však brzy setkal s kritikou a byl nahrazen novými poznatky od dalších teoretiků.
% Newton, Maxwell, Young, Fresnel, Goethe - zkoumání barev (goethe1840)
% https://books.google.cz/books?hl=en&lr=&id=mujNEAAAQBAJ&oi=fnd&pg=PP13&dq=colour+definition&ots=ECMtqQBf1F&sig=-kCLAYmN6owutngTKMQAJ4Fu2fc&redir_esc=y#v=onepage&q&f=false
% https://d1wqtxts1xzle7.cloudfront.net/112207887/jaic_v32_01-libre.pdf?1709866938=&response-content-disposition=inline%3B+filename%3DColour_theory_Definition_fields_and_inte.pdf&Expires=1752252945&Signature=ZfOn9uc-IsbJP3TrC5MvZV4lSsAOte25rIBlngEMk7LidfCjxXVDtY9tmgN8RPonzWEm4AD0TRa-vkG0SGxMuVksHhNWawTge-7yJNuDxPnS5vOLtoCdWxS3Qpb1Uj1qdG~UnhJq8XXZFIubw9iDNmgUWR0fhSkD~ejasDRYdj-~ZjkHxcYEVLaazY~OFY3QhmTVcZ25B1bzPfnxpwZjDOMPBVPJX397~6qY~0ZsyctLy0jzs8ehTZahxoXmJz8vCDESfASXwe~4IDfXkmWs-2u1JjHVFVDsWb-HVVp~witzDU-VBO0odEnlIta0~WRoEeWiA6oyjz893LdCmRO8iw__&Key-Pair-Id=APKAJLOHF5GGSLRBV4ZA
% další research pro historii a vývoj názorů, odkud a co je barva 
% https://www.pantone.com/articles/color-fundamentals/what-is-color#:~:text=Color%20is%20defined%20as%20the,others%20are%20absorbed%20by%20it.
% pantone stránka - moderní definice barvy, dnešní chápání 

\subsubsection{Vlastnosti barev}
% odstín, sytost a jas - https://dspace.cuni.cz/bitstream/handle/20.500.11956/188964/130380558.pdf?sequence=1
% odstín - název barvy
% sytost - množství šířeného světla, jak moc obsahuje barva bílou
% jas - intenzita barvy, kolik světla vychází z barvy

\subsection{Oko a barevné vidění}
% https://is.muni.cz/th/gpxge/Diplomova_prace.pdf
% struktura oka, tyčinky - černobílé, čípky - barevné vidění, proces předávání informace do mozku
% https://www.mendeley.com/catalogue/a9dda4be-621f-3d6f-b83d-343556b66070/
% trichromatické vidění - long(reddish), medium (greenish), and short (bluish) vawelenghts
% rozpoznávání barev v průběhu života
% https://dspace.cvut.cz/bitstream/handle/10467/91222/FBMI-BP-2020-Soukup-Jan-prace.pdf?sequence=-1&isAllowed=y
% teorie vnímání barev,  Young-Helmholtzova - tři typy čípků, Heringova teorie - proces
% spojení obou teorií - Dvoustupňová

\subsubsection{Vliv kontextu na vnímání barev}
% https://www.annualreviews.org/docserver/fulltext/vision/4/1/annurev-vision-091517-034231.pdf?expires=1752399604&id=id&accname=guest&checksum=25D512DF5FD9FEB5457EB71FA73304F5
% memory color effects - objekty spojené s určitou barvou (banán - žlutá) ovlivňují vnímání - objekt na šedém spektru vnímán jinak
% The empirical basis of color perception - https://www.sciencedirect.com/science/article/pii/S1053810002000144
% kontrast - pozadí barev, jiné vnímání
% věk - https://srcd.onlinelibrary.wiley.com/doi/full/10.1111/cdep.12447
% novorozeňata - horší barevné rozlišování, menší saturace

\subsubsection{Vliv barev v běžném světě}
% maybe not? 
% nálada, sociální postavení, aktuální stav: vzrušení, klid, stres, i chut
% https://pubs.acs.org/doi/abs/10.1021/es301685g - test zelené
% https://link.springer.com/chapter/10.1007/978-3-319-91716-0_32 - ovlivnovani barev na chovani zakaznika

\subsection{Poruchy barevného vidění}
%file:///C:/Users/sarka/Downloads/Barevne_videni_Archive.pdf
% dichromati, monochromati 
% prot-, deuter-, tri- (vady dle barvy - červená, zelená, modrá) - anopie vs anomálie
% katarakta (šedý zákal)
% Achromatopsie - poškození mozkových center
% chromatopsie - nepřirozeně barevné vidění, užíváním drog

\subsection{Teorie barev}
% https://app.knovel.com/kn/resources/kpCDTAE004/toc - historie
% teorií existuje mnoho, zmínit nejznámější:
% Aristoteles
% Isaac Newton
% Goethe
% Tobias Mayer
% JOHANNES ITTEN
% JOSEF ALBERS
%https://library.si.edu/exhibition/color-in-a-new-light/science#:~:text=Aristotle%20developed%20the%20first%20known%20theory%20of,until%20being%20replaced%20by%20those%20of%20Newton.
%https://web.mit.edu/22.51/www/Extras/color_theory/color.html

