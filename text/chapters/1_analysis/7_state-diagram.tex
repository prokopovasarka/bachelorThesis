\section{Stavový diagram}
Stavový diagram patří mezi využívané nástroje objektového modelování. Slouží k vizualizaci objektů
v rámci jeho životního cyklu, tedy stavů, kterými prochází, a událostí způsobujících změnu stavu~\cite{uml_stavovy_diagram}. Pomocí stavového
diagramu lze přesně popsat logiku chování systému napříč všemi případy užití~\cite{franek_objektove_metody_modelovani_5}. Komponentami diagramu jsou \textbf{Stav, Přechod stavu a Událost}~\cite{uml_stavovy_diagram}.

\begin{figure}[!ht]
    \centering
    \includegraphics[width=\paperwidth, angle=90]{images/state_diagram.png}
    \caption{Stavový diagram aplikace ColorLensAR, dle~\cite{uml_stavovy_diagram}.}
    \label{fig:State_diagram}
\end{figure}

\subsection{Stav}
Stav popisuje stav v životním cyklu objektů či interakce, ve které se stav nachází. Je vyznačena čtvercem se zaoblenými rohy
a popsána názvem stavu. Může být doplněna o další vnitřní stavy, které může stav nabývat. Příkladem stavu pro objekt Student je:
přihlášený, přijatý, zapsaný, apod. Speciálním případem stavů je počáteční a koncový stav.~\cite{uml_stavovy_diagram}

Ve stavovém diagramu pro vyvíjenou aplikaci v této práci viditelným na \ref{fig:State_diagram}, který popisuje nejdůležitější funkcionality aplikace,
definujeme následující stavy:

\paragraph{START}
Počáteční stav popisující první spuštění aplikace.
\paragraph{HELP}
Zobrazení nápovědy pro používání aplikace při přrvním spuštění aplikace.
\paragraph{MAIN}
Obrazovka hlavního menu.
\paragraph{AR\_CAMERA, PHOTO}
Stavy, ve kterých může uživatel vybírat barvu, buď z fyzického prostoru pomocí AR nebo z obrázku z galerie.
\paragraph{AR\_COLOR, PHOTO\_COLOR}
Stav popisující moment detekování barvy a výpis informací o barvě v boxu, přesněji název a kód.
\paragraph{GALLERY}
Stav, při němž probíhá výběr fotografie či obrázku z galerie fotoaparátu.
\paragraph{COLOR\_DETAIL}
Obrazovka s detailními informacemi o zvolené barvě. Pozadí je určené dle zvolené barvy, na obrazovce se nachází název barvy, HSV, CMYK a HEX kód a
monochromatická, komplementární a analogická paleta.
\paragraph{LIBRARY}
Obrazovka s knihovnou uložených barev, zobrazuje uložené barvy v orámovaném čtverci se zaoblenými rohy.
\paragraph{MULTI\_SELECT}
Stav, při kterém uživatel vybral v knihovně barev jednu či více barev.

\subsection{Přechod stavů}
Přechod stavů popisuje vztah mezi dvěma stavy, přesněji přechod z prvního stavu do druhého. Změna stavu proběhne při splnění podmínky a
značí se šipkou z jednoho stavu do druhého a jeho popisem.

\subsection{Událost}
Událost se stane v určitém časovém okamžiku, nemá trvání, například po uplynutí času či splnění podmínky.

Ve stavovém diagramu pro vyvíjenou aplikaci v této práci viditelným na \ref{fig:State_diagram}, který popisuje nejdůležitější funkcionality aplikace,
definujeme následující události v přechodech stavů:

\paragraph{První spuštění aplikace}
Přechod z počátečního stavu do stavu START při prvním spuštění aplikace.

\paragraph{Klik na \{název prvku\}}
Přechod popisuje akci kliknutí na prvek v aplikace. Prvkem může být tlačítko, ikona, text, zpětné tlačítko či čtverec s barvou v knihovně.

\paragraph{Vybrat fotografii/obrázek z galerie}
Přechod z GALLERY do PHOTO po vybrání obrázku či fotografie z galerie fotoaparátu.

\paragraph{Klik v prostoru}
Kliknutí ve stavu AR\_CAMERA na místo v prostoru pro získání informace o barvě.

\paragraph{Dlouhé podržení}
Přechod z LIBRARY do MULTI\_SELECT dlouhým přidržením jedné z barev.
