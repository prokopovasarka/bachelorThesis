\section{Barevné systémy a tvorba barevných palet}
%https://dspace.cuni.cz/bitstream/handle/20.500.11956/188964/130380558.pdf?sequence=1
% krátká zmínka o historických paletách - jen o jejich existenci a posunutí oproti těm od goetheho a Newtona
% barevne system, reprezentace (aditivni apod) RGB, CMYK - https://naos-be.zcu.cz/server/api/core/bitstreams/830d5224-cbd9-42c3-af0d-3ff9dd034966/content
% https://www.premocz.eu/barvy-rgb-cmyk-a-pantone
% https://www.shutterstock.com/blog/color-scheme-definitions-types-examples
V kapitole Barevné systémy a tvorba barevných palet si představíme základní používané palety,
se kterými se lze setkat nejen v uměleckém odvětví, ale také v každodenním životě. Různé barevné prostory
se hodí k rozdílným účelům. Jeden je vhodný pro zobrazení na monitoru počítače, další je lepší z hlediska výsledků tisku.

\subsection{Míchání barev}
Existují dva základní přístupy pro získávání barev, a sice aditivní (součtové) a substraktivní (odčítací).
Jejich primární rozdíl je zdroj světla, kdy v principu aditivním se jedná o míchání světla přímo ze zdroje, 
substraktivní princip naopak pracuje s odrazem světla od povrchu.~\cite{Mikolasova2014}

\paragraph{Aditivní míchání}\mbox{}\\
Při aditivním míchání se sčítají různě barevná světla s různou intenzitou. Primárními barvami jsou červená, modrá a zelená. 
Tímto principem lze získat veškeré barvy viditelné části spektra, při smíchání všech světel se stejnou intenzitou dosáhnete
bílé barvy. Při jejich úplné absenci je výsledkem černá.~\cite{Mikolasova2014}

\paragraph{Substraktivní míchání}\mbox{}\\
Substraktivní princip vychází z práce malířů s barvami. Každý pigment pohlcuje i odráží určitou
část světla a při odrazu dochází k odečítání jednotlivých složek. Čím více tedy barevných pigmentů přidáme,
tím tmavší výsledek získáme. Základem je bílá barva, ke které se nanášejí další. Vzájemným překrýváním
primárních barev získáváme další nové barvy. Zde jsou primárními barvami azurová, purpurová a žlutá.~\cite{Mikolasova2014}

\begin{figure}[!ht]
    \centering
    \includegraphics[width=0.9\linewidth]{images/adit_subtrak_barvy.jpg}
    \caption{Ukázka aditivních a substraktivních barev.~\cite{muni_mineralogie_barvy_aditivni}}
    \label{fig:Aditivni a Substraktivni}
\end{figure}

\subsection{Základní barevné systémy}
V této sekci si představíme základní systémy, kterými lze vyjádřit určité množství barev. Ačkoliv jich existuje
více, představíme si pouze ty relevantní pro naši práci, a to RGB, CMYK a HSV. Poslední část se bude věnovat také
systému Pantone.

\subsubsection{RGB}
RGB systém je jedním z nejznámějších systému, které vychází z vnímání barev lidským okem. Jeho název vychází z názvů základních složek,
červená (R), zelená (G) a modrá (B). Řadí se mezi aditivní systémy a je využit primáně pro obrazovky počítačů,
smartphonů a dalších monitorů. Každá složka disponuje 256 stupni intenzity (0-255), tedy pro získání černé je R, G i B rovno 0, naopak
pro bílou je nutné, aby R, G i B bylo rovno 255. Pro získání například šedé můžeme R, G i B nastavit na 100.~\cite{Mikolasova2014}

Problém nastává při tisku. Barvy míchá rozdílným způsobem, a tedy při získání informace od počítače je třeba pracovat s odlišným systémem~\cite{premocz}.
Ze systému RGB vychází další prostory vytvořené společnostmi pro lepší zpracování tiskárnami. Adobe RGB vytváří většinu barev, které zvládne CMYK, avšak při použití
RGB. Je tedy schopen získat přibližně 50 \% všech viditelných barev. Dále existuje například ProPhoto RGB od společnosti Kodak s větším gamutem, či sRGB od firem
Microsoft a Hewlett-Packard. sRGB je standardní paletou pro HTML a je hojně využíván v oblasti fotografií~\cite{Mikolasova2014}.

\subsubsection{CMYK}
CMYK využívá oproti RGB substraktivní metodu míchání a název vychází z barev nejčastěji využívaných tiskárnami: azurová (Cyan), purpurová (Magenta),
žlutá(Yellow) a černá (Značena jako "Key" doplňující kontrast). Dokáže však vyproduktovat pouze omezené množství barev, proto se také nabízejí barvy do tiskáren
s vlastními odstíny. Oproti monitoru není výsledek tak jasný a barevný.~\cite{premocz}

\subsubsection{HSV}
Model HSV je specifikován třemi hodnotami. Barevný tón (Hue) definovaný ve stupních, sytost (Saturation), která určuje odstín, a jas (Value) specifikující 
hodnotu bílého světla. Systém je využíván ke konkrétnějšímu popisu barev a velikou výhodou je jeho nezávislost na zařízení. Naopak nevýhodou je neplynulý barevný přechod.~\cite{Mikolasova2014}

\subsubsection{Pantone}
Pro jednotnost barev firem a společnostní, ať už je tisknuté kdekoliv, slouží Pantone. Jedná se o
standardizovaný vzorník barev, jenž vznikl v roce 1963. Obsahuje 1867 barev včetně metalických a reflexních odstínů. Vychází ze 16 barev, z nichž jsou přesně odměřeny
poměry barev pro objektivní výsledek.~\cite{premocz}

\begin{figure}[!ht]
    \centering
    \includegraphics[width=0.8\linewidth]{images/pantone.jpg}
    \caption{Pantone vzorník.~\cite{premocz}}
    \label{fig:Pantone}
\end{figure}