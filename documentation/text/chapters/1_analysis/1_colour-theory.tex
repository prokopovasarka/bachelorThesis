\section{Teorie barev}
Teorie barev by se dala popsat jako soubor postupů a pravidel zahrnující 
používání primárních pigmentových barev, vytváření barevné harmonie, míchání barev
a jejich aplikaci.

Abychom se však mohli věnovat samotným teoriím a určit postupy, se kterými budeme k této práci
přistupovat, je třeba začít s úplnmými základy a definovat si, co vlastně je barva a jak je vnímána
lidským okem.


\subsection{Barva}
Vnímání barev je velmi subjektivní a již od dob Aristotela nám jejich zkoumání nepřineslo
jednotný názor, který by přesně definoval barvu.
Ten založil své poznatky na pozorování slunečního světla, které při odrazu či průchodu objektem
snižuje svou intenzitu nebo je ztmaveno. Vnímal tak barvu jako mísení, míchání, superpozici či juxtapozici
černé a bílé \cite{goethe1840}.